\documentclass[12pt]{article}
\usepackage{cv} %use xelatex
\title{Curriculum Vitae}
\author{Andrew J.\ Zimolzak}
\begin{document}

\thispagestyle{fancy}

\section*{Personal Information}

\textbf{Home address:}\\
***REMOVED*** Street, ***REMOVED***\\
***REMOVED***, MA ***REMOVED***\\
***REMOVED*** (mobile)\\
***REMOVED***@gmail.com

\textbf{Research address:}\\
***REMOVED***\\
Boston, MA ***REMOVED***\\
***REMOVED***\\
***REMOVED*** (fax)

\textbf{Clinical practice address:}\\
133 ***REMOVED*** Avenue\\
Boston, MA ***REMOVED***-3904\\
***REMOVED***

\textbf{Birth date:} ***REMOVED***\\
\textbf{Birthplace:} ***REMOVED***

\section*{Education}

\textbf{Michigan State University.} 8/13/1998\ndash5/3/2002.\\
150 Administration Building; East Lansing, MI 48824-0210.\\
BS in Biochemistry with High Honor.

\textbf{Washington University in St.\ Louis School of Medicine.}
8/19/2002\ndash\linebreak[0]5/6/\linebreak[0]2007.\\
Campus Box 8021, 660 S.\ Euclid; St.\ Louis, MO 63110.\\
MD.

\textbf{University of Missouri-Columbia.} 7/1/2007\ndash6/30/2008.\\
MA419 Medical Sciences Building DC043.00; Columbia, MO 65212.\\
Internal medicine PGY 1.

\textbf{Saint Louis University Hospital.} 7/1/2008\ndash6/30/2010.\\
3635 Vista Avenue, 14th Floor S, Desloge Towers; St.\ Louis MO
63110-0250.\\
Internal medicine PGY 2 and 3.

\textbf{Harvard Medical School.} 7/1/2011\ndash5/30/2013.\\
25 Shattuck Street; Boston, MA 02115.\\
MMSc in Medical Informatics.

\newpage

\section*{Professional Experience} % reverse chronol.

\textbf{Research fellow,} Children’s Hospital Informatics Program,
Intelligent Health Lab\-o\-ra\-to\-ry,
\linebreak[0]7/1/\linebreak[0]2011\ndash{}\linebreak[0]pres\-ent.
Analyze large pharmacy claims databases to develop predictive models
of medication adherence. Use SAS software to operate on the primary
data set from Aetna, Inc., which selected commercial insurance members
with hyperlipidemia and comprised 3 major database tables with 61
million enrollment records, 200 million medical claims records, and 90
million prescription claims records. Secondary data sets included
laboratory and survey data from Aetna, and data from a different
insurer on different chronic conditions. Perform modeling using
techniques of multivariable logistic regression and multivariable
adaptive regression splines. Present at and attend regular laboratory
meetings and journal clubs. Coursework including Harvard School of
Public Health Program in Clinical Effectiveness (introductory
epidemiology, biostatistics, health services research,
pharmacoepidemiology); and coursework in biomedical informatics, data
mining \& statistical machine learning, population health informatics,
business \& management, and grant writing. Attend American Medical
Informatics Association symposia and National Library of Medicine
informatics conferences. Professor: Kenneth Mandl, Harvard Medical
School. Funded by National Library of Medicine under grant number
5-T15-LM007092-20, to Boston Informatics Training Program. Program
director: Alexa McCray, Harvard Medical School.

\textbf{Urgent care physician,} Harvard Vanguard Medical associates,
10/15/2011\ndash{}ongoing. Diagnose and treat minor acute medical problems,
including abscesses, simple lacerations, pelvic conditions, and
initial management of fractures. Supervise and consult with nurse
practitioners and physician assistants as needed. Interpret urgent
radiographs, electrocardiograms, and laboratory tests. Our clinic
capabilities include intravenous fluids and intravenous antibiotics.
Supervisor: David Meenan, 133 ***REMOVED*** Ave, Boston MA ***REMOVED***.

\textbf{Fellowship liaison,} Harvard Medical School,
6/2012\ndash5/2013. Coordinate the twice-week\-ly, two-hour
informatics seminar, a for-credit course that 20 students attend.
Assist with recruiting and scheduling 25 different speakers from
Harvard and externally. Assure group consensus about the direction of
this fellow-run seminar.

\textbf{Advisor,} Medication Adherence Request for Proposals, Aetna, Inc.,
1/2013\ndash{}7/2013. Review competitive proposals/bids from 7 vendors to
Aetna, Inc., for a contract to identify modifiable risk factors for
medication nonadherence. Provide expertise on statistical modeling
techniques and on published literature about medication adherence.

\textbf{Group meeting coordinator,} Children’s Hospital Informatics
Program, Intel\-li\-gent Health Lab\-o\-ra\-to\-ry,
8/1/\linebreak[0]2011\ndash{}\linebreak[0]5/1/2013. Schedule presenters
from approximately 20 researchers attending the laboratory meeting, 1
hour meeting every 1\ndash2 weeks.

\textbf{Early career physician council,} Massachusetts American
College of Physicians,
9/\linebreak[0]2011\ndash{}\linebreak[0]on\-go\-ing. Organize activities
for early career internists in Massachusetts, as part of an
approximately 15-person council. Sit on planning committee for
continuing medical education events, coordinate social events.

\textbf{Internal medicine chief resident,} Saint Louis University
Hospital, 7/1/2010\ndash\linebreak[0]6/30/\linebreak[0]2011. Included academic
appointment as instructor in internal medicine and hospital
appointment as physician. Attended on inpatient \& outpatient general
medicine services in university and VA hospitals. Accepted transfers,
supervised and improved patient handoffs. Precepted medical students
and ran resident educational conferences. Created schedules for 130
residents and rotators, monitored duty hours, managed high volume
e-mail account. Interviewed residency program applicants and sat on
selection committee. Program director: Miguel Paniagua, Saint Louis
University.

\textbf{Patient acceptance, placement, and discharge committee,} Saint
Louis University Hospital, 5/2010\ndash\linebreak[0]6/2011. Identified
and reduced systematic features that lead to hospital gridlock.
Membership included hospital executives and department managers.

\textbf{Chief resident meeting,} 4/26\ndash27, 2010. Alliance for
Academic Internal Medicine / Association of Program Directors in
Internal Medicine national meeting, Baltimore MD. National workshop on
education and leadership prior to the chief resident year.

\textbf{Research elective,} 2/2007\ndash4/2007. Analyzed associations
in a database of 12,000 outpatients associated with 40,000 diagnoses.
Wrote Perl code to find associations between any given pair of
diseases in a patient, and to output the strongest associations in an
easily visualized format. Professor: Walton Sumner, Washington
University in St.\ Louis.

\textbf{Professorial assistant,} 9/1998\ndash5/2002. Used molecular
biology and cell culture techniques to analyze factors leading to
malignant transformation of human cells. Professor: J.\ Justin
McCormick, Michigan State University.

\textbf{Computer skills.} Advanced experience with Microsoft Windows, Word,
Excel, and PowerPoint, and with Macintosh OS X. Intermediate
experience with the Perl programming language, SAS and R statistical
programming, Unix shell commands and shell programming, Git source
code management, GNU Emacs, \LaTeX, SQL, Apple Final Cut video editing,
HTML4, CSS, and Adobe Photoshop; and with Cerner PowerChart, Epic
Hyperspace, and CPRS/VistA electronic medical records systems.

\section*{Credentials and Memberships}

\textbf{ABIM} certified 8/10/2011, valid through 12/31/2021.
First-time test taker, initial certification.\\
\textbf{Massachusetts} full license 6/2011\ndash{}present.\\
\textbf{Missouri} permanent medical license 6/2010\ndash{}present.\\
\textbf{American College of Physicians} member 2010\ndash{}present.\\
\textbf{USMLE} passed Step 1 6/2005, Step 2 CS 4/2007, Step 2 CK 4/2007, Step 3
5/2010.\\
\textbf{ACLS} last certification 9/24/2012, recommended renewal
9/2014. Basic CPR certified 5/2011.

\section*{Honors and Awards}

\textbf{Third prize, patient safety \& quality improvement category,} 8/2010.
American College of Physicians, Missouri Chapter Scientific Meeting,
associate poster competition.

\textbf{Biochemistry Undergraduate Research Fellowship,} Michigan State
University, 2000.

\textbf{Tower Guard,} 1999\ndash2000. Service-oriented academic honor
society for Michigan State University sophomores.

\textbf{Distinguished Freshman Scholarship,} Michigan State
University. Competitive four-year full tuition scholarship awarded to
35 students out of the incoming class of about 10,000.

\textbf{Valedictorian} of Lumen Christi High School, 1998. Graduating
class of about 180 students.

\section*{Outside Interests}

\textbf{SLU Literature and Medicine Group,} member in 2008 and 2010.\\
\textbf{Creative writing,} National Novel Writing Month, winner 2005, 2006,
2008, 2009, 2010, 2011; Washington University Medical School Write Club
member, 2003\ndash2007.\\
\textbf{Musical} cast member of \emph{Guys and Dolls} 2006, and
\emph{Damn Yankees!}\ 2007.\\
\textbf{Class show} technical director, spring 2003, 2004, and 2006.\\
\textbf{Additional interests} include travel (Stockholm, Copenhagen, Avignon,
Krak\'ow), music (organ works, pop music of They Might Be Giants), black
and white photography, social dance, and bread baking.

\section*{Presentations/Works}
Zimolzak AJ, Spettell CM, Fernandes J, Fusaro VA, Palmer NP, Saria S,
Kohane IS, Jonikas M, Mandl KD. (2013) \textbf{Early Detection of Poor
  Adherers to Statins: Applying Individualized Surveillance to Pay for
  Performance.} PLoS ONE 8(11):e79611. \\
doi:10.1371/journal.pone.0079611

Zimolzak AJ. \textbf{Medication Adherence: How Should We Measure It, and Can
We Detect It Early?} Harvard Medical School, Master of Medical Science
thesis, 5/17/2013. Thesis committee members: Kenneth D.\ Mandl, MD,
MPH; Sebastian Schneeweiss, MD, SD; Aurel Cami, PhD.

Zimolzak AJ, Spettell CM, Fernandes J, Fusaro VA, Palmer NP, Saria S,
Jonikas M, Kohane IS, Mandl KD. \textbf{Early Statin
  Adherence As a Predictor of Later Adherence.} American Medical
Informatics Association Summit on Clinical Research Informatics. San
Francisco; 3/20/2013.

Adeimy C, Zimolzak AJ, Paniagua MA. \textbf{Improved Supervision of
  Transfers of Care and Education for Night Residents.} American
College of Physicians, Missouri Chapter Scientific Meeting, associate
poster competition, 8/2010.

\subsection*{Fellow presentations (selected) 2011\ndash2013:}
Predictive Modeling of ICU Discharge, journal club. 11/18/2011.\\
Intro to Data Mining with Regression Splines in R. 9/6/2012.

\subsection*{Chief resident presentations 2010\ndash2011:}
Chest Radiology Cases. 7/29/2010.\\
Clinicopathologic Conference: 51 year-old woman with 3 weeks of fever.
10/20/2010.\\
Clinicopathologic Conference: 81 year-old woman with fatigue and
melena. 2/16/2011.\\
Resident report moderator (30 minute conference, approximately 100
conferences yearly).\\
Quality Improvement (six different one-hour morbidity \& mortality
conferences).

\subsection*{Resident presentations 2007\ndash2010:}
Case Presentation: 75 year-old man with leg swelling. 10/10/2007.\\
Case Presentation: 79 year-old man with fatigue. 12/5/2007.\\
Prostate Cancer Screening and Mortality. 5/11/2009.\\
Journal Club: Results of the NICE-SUGAR Study. 10/22/2009.\\
Evidence-Based Medicine: \emph{Clostridium difficile}. 11/11/2009.\\
Case Presentation: 24 year-old with cystic fibrosis and 4 months of
dyspnea. 1/5/2010.\\
Aspirin for Primary Cardiovascular Prophylaxis. 2/8/2010.\\
Electronic Reminder Systems and Why We Ignore Them. 2/15/2010.

\section*{References}
Kenneth Mandl. 300 Longwood Ave; Boston MA 02115. ***REMOVED***.\\
David Meenan. 133 ***REMOVED*** Ave; Boston MA ***REMOVED***. ***REMOVED***.\\
Alexa McCray. 10 Shattuck St; Boston MA 02115. ***REMOVED***.\\
Miguel Paniagua. 1402 S.\ Grand, 14S FDT; St.\ Louis MO 63104.
***REMOVED***.\\
H. Douglas Walden. 1402 S.\ Grand, 12S FDT; St.\ Louis MO 63104.
***REMOVED***.\\
Walton Sumner. 660 S.\ Euclid, Campus Box 8005; St.\ Louis MO 63110.
***REMOVED***.

\end{document}

% LocalWords:  Desloge Shattuck MMSc ent Mandl LM radiographs Meenan Advisor KD
% LocalWords:  handoffs Precepted rotators Paniagua Lumen Schneeweiss Aurel FDT
% LocalWords:  Cami Spettell Fernandes Fusaro Saria Jonikas Kohane Adeimy ra ry
% LocalWords:  Clostridium difficile Clinicopathologic Longwood ly li ing Krak
