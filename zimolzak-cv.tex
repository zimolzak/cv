\documentclass[11pt]{report}
\usepackage{cv}
\begin{document}

\textbf{Specific Aims}

Medication nonadherence leads to higher mortality, worse disease
prognosis, and higher healthcare costs (Osterberg and Blaschke, 2005).
Excess costs have been estimated at \$289 billion annually (New
England Healthcare Institute, 2009). Interventions that improve
adherence have been developed, but research also needs to address an
important opportunity regarding the definition of adherence itself.
Many studies calculate a measure called proportion of days covered
(PDC) and use PDC < 80\% to define non-adherence to any drug, but
because half-lives and clinical effects differ from drug to drug, it
is highly unlikely that an 80\% threshold could be defended for all
drugs; research commentary (Steiner, 2012) agrees that this threshold
is arbitrary. I will develop and validate data-driven adherence
measures that are critically needed in adherence research.

Many interventions to improve adherence are complex or costly (Haynes
\emph{et al.}, 2008). For example, several pharmacists, a
cardiologist, a geriatrician, a behavioral scientist, and a cognitive
psychologist trained an intervention pharmacist, who intervened over a
9-month period; and effects dissipated in 3 months (Murray \emph{et
  al.}, 2007). Due to the large amount of resources required for such
an intervention, prior studies have constructed predictive statistical
models of adherence, to target interventions to patients who would
benefit the most. Adherence models based on sociodemographic, payment,
and comorbidity variables have not proven useful in practice, though,
and it can be concluded that adherence must be measured, not inferred
from such variables (Steiner, 2012). Our \underline{preliminary work},
performed on an Aetna, Inc.\ dataset of 600,000 patients with
hyperlipidemia, corroborates this observation, but we show in addition
that an important signal exists in prescription claims: specifically,
early patterns of statin prescription refills \underline{are highly
  predictive} of future statin adherence. In Aims 2 and 3 of the
proposed work, we will extend our work on statins to many more drug
classes and determine whether adherence to current medications
predicts adherence to new medications.

This innovative work will be conducted in a environment that is unique
because of our access not only to the hyperlipidemia claims dataset,
but to an unusual wealth of laboratory data, an 8.5-million-member
dataset from a different insurer, and drug databases that we have used
in predictive research. The proposed study will define what are
\underline{clinically meaningful} adherence levels for
\underline{multiple drug classes} and improve \underline{early
  detection} of those unlikely to achieve meaningful adherence.

\textbf{Aim 1. We will derive data-driven thresholds for adherence
  research that relate prescription fills to meaningful outcomes.} We
will determine which adherence thresholds are related to laboratory
outcomes. For patients on statins or oral antidiabetic agents, we will
link prescription fills with cholesterol or glycated hemoglobin
levels. Using segmented regression, we will determine the most
clinically meaningful adherence threshold (80\% or otherwise) that
relates to change in cholesterol or glycated hemoglobin over time, and
we will test the hypothesis that a threshold exists. We will test the
hypothesis that dose-response curves will vary by age and renal
function. Aim 1 will change adherence research methodology, defining a
more relevant adherence threshold to be used in research, and it will
change policy, determining what method government quality assessment
programs should use to rate insurance plans on members' adherence.

\textbf{Aim 2. We will generalize our early detection model of statin
  adherence to additional drug classes.} Through our collaboration
with Aetna, we now have access not only to the 600,000-member
hyperlipidemia dataset, but also to medical and pharmacy claims, plus
laboratory data, on all commercial insurance members for 4 years. We
will apply the early detection algorithm from our initial statin
adherence study to 12 chronic medication classes to generalize the
algorithm. We can also derive data-driven adherence measures for drug
classes from known pharmacodynamic characteristics of drugs, for
medication classes for which outcomes are not readily available in
insurance claims data. This aim will improve management of patients
with chronic conditions by furnishing models for early detection of
patients prone to clinically meaningful decreases in adherence, who
would therefore benefit from a possibly complex or costly
intervention.

% 1 and 2 together?

\textbf{Aim 3. We will develop a model to accurately predict adherence
  before new drug initiation.} We will use these novel predictors of
adherence to newly initiated medications: adherence to current
medications; subjects' renal function; known adverse reactions,
pharmacodynamics, and kinetics (Lexi-Comp); known drug-drug
interactions (VantageRx); and predicted adverse reactions and
interactions (Cami \emph{et al.}, 2011, 2013). We will correlate the
novel predictors with adherence to newly initiated statins,
antihypertensives, or oral antidiabetic agents. This study will also
allow us to determine what drug histories best predict adherence to
these three drug classes. We hypothesize that adherence to chronic
medications (e.g.\ antidepressants) will correlate positively with
adherence to all three classes, and that increase in adverse event
severity \& likelihood will correlate negatively with adherence. Aim 3
will produce an improved predictive model that organizations can apply
to a large population, to identify individuals who would benefit from
interventions that will improve their adherence and ultimately
decrease mortality, disease complications, and healthcare costs.

\end{document}

% LocalWords:  healthcare Osterberg Blaschke statins antihypertensives Aetna JF
% LocalWords:  comorbidity sociodemographic comorbidities dataset statin PDC
% LocalWords:  hyperlipidemia pharmacoepidemology datasets biometric Biomedical
% LocalWords:  antidiabetic pharmacodynamic Lexi VantageRx Cerner Multum Cami
% LocalWords:  pharmacodynamics
% LocalWords:  antihypertensive nonadherent glycated Engl Ackloo Sahota Yao
% LocalWords:  Cochrane Syst
