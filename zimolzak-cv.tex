\documentclass[10pt]{article}
\usepackage{cv} %use xelatex
\title{Curriculum Vitae}
\author{Andrew J.\ Zimolzak}
\begin{document}

\thispagestyle{fancy}

\section*{Personal Information}

\begin{tabular}{l l}
\textbf{Home address:}  & \textbf{Research/clinical address:}\\
2202 Macarthur St       & 2450 Holcombe, suite 01Y\\
Houston, TX 77030       & Houston, TX 77021\\
517-740-7761 (mobile)   & 713-791-1414 ext.\ 10226\\
zimolzak@fastmail.com   & Andrew.Zimolzak@bcm.edu\\
\\
\textbf{Birth date:} November 21, 1980\\
\textbf{Birthplace:} Alma, MI, U.S.A.
\end{tabular}

\section*{Education}

\textbf{Michigan State University.} 8/13/1998\ndash5/3/2002.\\
150 Administration Building; East Lansing, MI 48824-0210.\\
BS in Biochemistry with High Honor. Member of the Honors College.

\textbf{Washington University in St.\ Louis School of Medicine.}
8/19/2002\ndash\linebreak[0]5/6/\linebreak[0]2007.\\
Campus Box 8021, 660 S.\ Euclid; St.\ Louis, MO 63110.\\
MD.

\textbf{University of Missouri-Columbia.} 7/1/2007\ndash6/30/2008.\\
MA419 Medical Sciences Building DC043.00; Columbia, MO 65212.\\
Internal medicine PGY 1.

\textbf{Saint Louis University Hospital.} 7/1/2008\ndash6/30/2010.\\
3635 Vista Avenue, 14th Floor S, Desloge Towers; St.\ Louis MO
63110-0250.\\
Internal medicine PGY 2 and 3.

\textbf{Harvard Medical School.} 7/1/2011\ndash5/30/2013.\\
25 Shattuck Street; Boston, MA 02115.\\
MMSc in Medical Informatics.

\section*{Professional Experience} % reverse chronf

\textbf{Clinical informatician} at the VA Center for Innovations in
Quality, Effectiveness and Safety (IQuESt), and at the Baylor
Institute for Clinical and Translational Research,
11/12/2018\ndash{}present. \textbf{Assistant professor,} Baylor
College of Medicine. I am a clinical subject matter expert who reviews
large collections of fragmented data elements to unify them into
clinically meaningful concepts, thus making diverse clinical research
studies possible.  These efforts are often termed \textbf{secondary
  use} of electronic health record data, or \textbf{clinical research
  informatics.} I interpret clinical language and systems for data
pull engineers/analysts, or vice versa (interpreting technical
language for clinical investigators). I access a data warehouse
covering twenty million unique individuals, and billions of
observations, from 1999 to the present, aggregated from 130 VA sites
(but incompletely standardized/harmonized). Major projects include
disseminating algorithms to detect missed laboratory test follow-ups,
and clinical trials matching. Supervisors: Hardeep Singh, Laura
Petersen, Chris Amos.

\textbf{Clinical informatician,} Massachusetts Veterans Epidemiology
Research and Information Center (MAVERIC),
10/20/\linebreak[0]2014\ndash{}11/9/2018. \textbf{Assistant
  professor,} Boston University School of Medicine,
8/2016\ndash{}11/2018, and reciprocal appointment as Lecturer, Harvard
Medical School. My work enabled a \textbf{genomic precision medicine}
platform and ultimately learning health system for cancer, as well as
a 13,000-subject point of care, randomized \textbf{comparative
  effectiveness trial} of two antihypertensives. I developed a system
to ensure \textbf{reusable phenotyping algorithms} for genomic
association and other studies, and I directly exported and encrypted
data for transfer to the mirror system adopted at Oak Ridge National
Laboratory. I am a site investigator and drafted a local IRB protocol
for a human subjects study involving \textbf{smartphone sensors for
  health monitoring}, sponsored by a grant from DARPA to
Lockheed-Martin. I have collaborated on writing and submission of a
U01 grant. I served as \textbf{co-program-director} of the Big
Data-Scientist Training Enhancement Program (BD-STEP) site in Boston,
and served as principal investigator on a study of genomic predictors
of stage of lung cancer at presentation. I mentored an internal
medicine resident, who helped me make a highly accurate
\textbf{natural language processing classifier} of serum protein
electrophoresis reports. Finally, I worked as a \textbf{teaching
  attending} on the inpatient medicine service, supervising house
staff and students from three residency programs, two medical schools,
and an interdisciplinary PhD program. Supervisors (research): Nhan Do,
Valmeek Kudesia, Mary Brophy, Louis Fiore. Clinical: Steven Simon, Jay
Orlander, Anthony Breu.

\textbf{Consultant,} OM1 outcomes measurement and analytics,
2/2017\ndash{}9/2017. Clinical subject matter expert assisting mainly
with harmonizing definitions of outcome measures from multiple
projects. Also advised on harmonizing clinical laboratory test units,
and ICD-10 codes or UMLS vocabularies relevant to certain data
elements. References: Vandana Menon, Michelle Leavy.

\textbf{Research collaborator,} Massachusetts Institute of Technology,
Laboratory for Computational Physiology,
12/6/2013\ndash{}\linebreak[0]9/2016. Lead clinician on a project to
develop a data-based definition of acute kidney injury, using records
of 11,000 patients from MIMIC II, an open collection of intensive care
unit data collected over 8 years. Collaborated on writing and
submission of an NIH R25 grant. Professor: Leo Celi, Harvard-MIT
Division of Health Science and Technology.

\textbf{Research fellow,} Children's Hospital Informatics Program,
Intelligent Health Lab\-o\-ra\-to\-ry,
\linebreak[0]7/1/\linebreak[0]2011\ndash{}\linebreak[0]1/31/2014.
Analyzed pharmacy claims to develop predictive models of medication
adherence. Used \textbf{SAS software and multivariable logistic
  regression} to analyze 61 million enrollment records, 200 million
medical claims, and 90 million prescription claims. I designed and
wrote all code to perform the analysis, resulting in 2,000 lines of
code in the final project. Other data and methods included laboratory
and health risk assessment data, multivariable adaptive regression
splines, and Fourier spectral analysis. Coursework in
\textbf{epidemiology, biostatistics, data mining, machine learning,
  and grant writing,} partially through Harvard School of Public
Health Program in Clinical Effectiveness. Professor: Kenneth Mandl,
Harvard Medical School. Program director: Alexa McCray, Harvard
Medical School.

\textbf{Urgent care physician,} Harvard Vanguard Medical Associates,
10/15/2011\ndash{}6/30/2018. Diagnose and treat patients with acute
medical problems, e.g.\ respiratory or gastrointestinal infections,
dehydration, sprains, abscesses, lacerations, pelvic conditions,
initial management of fractures. Supervise nurse practitioners and
physician assistants. Supervisor: David Meenan.

\textbf{Fellowship liaison,} Harvard Medical School,
6/2012\ndash5/2013. Coordinated a twice-week\-ly, two-hour informatics
seminar, a for-credit course that 20 students attend. Assisted with
recruiting and scheduling 25 different speakers from Harvard and
externally. Assured group consensus about the direction of this
fellow-run seminar.

\textbf{Internal medicine chief resident,} Saint Louis University
Hospital, 7/1/2010\ndash\linebreak[0]6/30/\linebreak[0]2011. Managed
and scheduled 130 residents. Academic appointment as instructor in
internal medicine and hospital appointment as physician. Attended on
inpatient \& outpatient general medicine in university and VA
hospitals. Accepted transfers, supervised and improved patient
handoffs. Precepted medical students, ran approximately 100 resident
report conferences, 6 morbidity and mortality conferences, and two
clinicopathologic conferences. Interviewed residency program
applicants and sat on selection committee. Program director: Miguel
Paniagua, Saint Louis University.

\textbf{Research elective,} 2/2007\ndash4/2007. Analyzed associations
in a database of 12,000 outpatients associated with 40,000 diagnoses.
Wrote \textbf{Perl code} to find associations between any given pair of
diseases in a patient, and to output the strongest associations in an
easily visualized format. Professor: Walton Sumner, Washington
University in St.\ Louis.

\textbf{Professorial assistant,} 9/1998\ndash5/2002. Used molecular
biology and cell culture techniques to analyze factors leading to
malignant transformation of human cells. Professor: J.\ Justin
McCormick, Michigan State University.

\textbf{Technology skills.} Experience with SAS and R statistical
programming, Git source code management, and SQL all since 2011; and
with the Python programming language since 2013. GitHub username:
zimolzak (have written code involving topics such as a Twitter text
generator, cryptography, satisfiability solving, celestial navigation,
and Amazon Web Services tasks). Experience with the Perl programming
language, Linux/UNIX, \LaTeX, and GNU Emacs since 2001. Intermediate
experience with audio, video, and photo equipment and
editing. \emph{Electronic medical records:} experience with Epic
Hyperspace since 2009, CPRS/VistA since 2007, and 1 year experience
with Cerner PowerChart.

\section*{Credentials and Memberships}
\textbf{American Board of Internal Medicine} certified 8/10/2011,
valid through 12/31/2021.\\
\textbf{Clinical Informatics} board-certified 1/1/2014, valid through
1/31/2024.\\
\textbf{Texas} full medical license 8/2018\ndash{}present. No.\ R8850,
exp 8/31/2020.\\
\textbf{Massachusetts} full medical license 6/2011\ndash{}present.
No.\ 249050, exp 11/21/2019.\\
\textbf{Missouri} full medical license 6/2010\ndash{}present. No.\ 2010020878, exp 1/31/2020\\
\textbf{American College of Physicians} member 2010\ndash{}present.\\
\textbf{American Medical Informatics Association} member
2011\ndash{}present.\\
\textbf{Early career physician council,} Massachusetts American
College of Physicians, 9/\linebreak[0]2011\ndash{}2015.\\
\textbf{USMLE} passed Step 1 6/2005, Step 2 CS 4/2007, Step 2 CK
4/2007, Step 3 5/2010.\\
\textbf{ACLS} and BLS last certification 11/9/2014.

\section*{Honors and Awards}

\textbf{Reviewers' Choice Abstract,} American Society of Human
Genetics Annual Meeting, 2017. Abstract entitled ``Developing
validated phenotypic cancer cohorts\ldots{}.''\\
\textbf{Finalist,} Surescripts 2014 Adherence Challenge. Member of a
team selected as one of ten finalists. Primary responsibility for
drafting the proposal. 8/15/2014\ndash{}1/8/2015.\\
\textbf{Third prize, patient safety \& quality improvement category,}
8/2010. American College of Physicians, Missouri Chapter Scientific
Meeting, associate poster competition.\\
\textbf{Tower Guard,} 1999\ndash2000. Service-oriented academic honor
society for Michigan State University sophomores.\\
\textbf{Distinguished Freshman Scholarship,} Michigan State
University. Competitive four-year full tuition scholarship awarded to
35 students out of the incoming class of about 10,000.

\section*{Outside Interests}

\textbf{Online courses} in machine learning (Coursera), theoretical
computer science (Udacity), and contract law (edX).\\
\textbf{Literature:} Book clubs with pediatrics residency \& SLU
Medical School. NaNoWriMo winner. WUMS Write Club.\\
\textbf{Class show} technical director, spring 2003, 2004, and 2006.\\
\textbf{Additional interests:} travel (Halifax, Stockholm, Avignon,
Krak\'ow), black and white photography and darkroom, Rubik's Cube,
sailing (Mercury 15$'$, Sonar 23$'$), ukulele.

\section*{Publications}

Vassy JL, Brunette CA, Majahalme N, Advani S, MacMullen L, Hau C,
Zimolzak AJ, Miller SJ. \textbf{The Integrating Pharmacogenetics in
  Clinical Care (I-PICC) Study: Protocol for a point-of-care
  randomized controlled trial of statin pharmacogenetics in primary
  care.} Contemp Clin Trials 75:40\ndash{}50.

Danziger J, Zimolzak AJ. \textbf{Residual Confounding Lurking in Big
  Data: A Source of Error.} In: MIT Critical Data, editors. Secondary
Analysis of Electronic Health Records. Cham, Switzerland: Springer;
2016. pp.\ 71\ndash{}78.

Fiore L, Ferguson RE, Brophy M, Kudesia V, Shannon C, Zimolzak A,
Pierce-Murray K, Turek S, Lavori P. (2016) \textbf{Implementation of a
  Precision Oncology Program as an Exemplar of a Learning Health Care
  System in the VA.} Fed Pract 33(suppl 1):S26\ndash{}S30.

Badawi O, Brennan T, Celi LA, Feng M, Ghassemi M, Ippolito A, Johnson
A, Mark RG, Mayaud L, Moody G, Moses C, Naumann T, Pimentel M, Pollard
TJ, Santos M, Stone DJ, Zimolzak A. (2014) \textbf{Making Big Data
  Useful for Health Care: A Summary of the Inaugural MIT Critical Data
  Conference.} JMIR Med Inform 2(2):e22.\\
doi:10.2196/medinform.3447

Celi LA, Zimolzak AJ, Stone DJ. (2014) \textbf{Dynamic Clinical Data
  Mining: Search Engine-Based Decision Support.} JMIR Med Inform
2(1):e13. doi:10.2196/medinform.3110

Zimolzak AJ, Spettell CM, Fernandes J, Fusaro VA, Palmer NP, Saria S,
Kohane IS, Jonikas M, Mandl KD. (2013) \textbf{Early Detection of Poor
  Adherers to Statins: Applying Individualized Surveillance to Pay for
  Performance.} PLoS ONE 8(11):e79611.
doi:10.1371/journal.pone.0079611

Zimolzak AJ. \textbf{Medication Adherence: How Should We Measure It,
  and Can We Detect It Early?} Harvard Medical School, Master of
Medical Science thesis, 5/17/2013.

\section*{Presentations} %reverse chron

Zimolzak A. \textbf{Early and late stage lung cancer: correlating
  clinical data and genomics in the Million Veteran Program.} American
Thoracic Society International Conference. Dallas; 5/20/2019
(accepted).

Do N, Bono J, Fillmore N, Zimolzak A, Johnson B, Meng F, Elbers D,
Hall R, Ajjarapu S, Brophy M, Elkin P. \textbf{Development of an AI
  empowered Electronic Molecular Tumor Board Application Connected
  Utilizing the SMART on FHIR Framework.} American Medical Informatics
Association Annual Symposium. San Francisco; 11/5/2018.

Advani S, Brunette CA, Miller SJ, Majahalme N, MacMullen L, Hau C,
Zimolzak AJ, Vassy JL. \textbf{The Integrating Pharmacogenetics in
  Clinical Care (I-PICC) study: Baseline characteristics of
  participants in a point-of-care randomized trial.} American Society
of Human Genetics Annual Meeting. San Diego; 10/17/2018.

Johnson BR, Elbers DC, Pierce E, Meng F, Fillmore N, Ayandeh S, Chen
D, Selva L, Begley NB, Zimolzak AJ, Brophy MT, Do NV. \textbf{The
  Veterans Health Administration's Research Precision Oncology
  Project: Integrating Real-World Data into a Learning Health System.}
American Society of Human Genetics Annual Meeting. San Diego;
10/18/2018.

Brunette CA, Miller SJ, Majahalme N, Hau C, MacMullen L, Advani S,
Zimolzak AJ, Vassy JL. \textbf{Pragmatism in Pharmacogenetics Trials:
  A PRECIS-2 perspective on the Integrating Pharmacogenetics in
  Clinical Care (I-PICC) Study.} American Society of Human Genetics
Annual Meeting. San Diego; 10/19/2018.

Do N, Pierce-Murray K, Peirce E, DeDomenico C, Meng F, Elbers D,
Katcher B, Hall R, Zimolzak A, Ajjarapu S, Shannon C, Turek S, Johnson
B, Fillmore N, Brophy M, Fiore L. \textbf{The Development of the
  Research Precision Oncology Program Data Repository (PODR) in the
  Veterans Affairs Healthcare System.} American Medical Informatics
Association Annual Symposium. Washington, D.C.; 11/6/2017.

Ajjarapu S, Meng F, Elbers D, Do N, Hall R, Pierce-Murray K, Selva L,
Katcher B, Johnson B, Zimolzak A, DeDomenico C, Brophy M, Fiore L.
\textbf{Releasing De-identified Clinical, Imaging, and Genomic Data
  from the VA to External Repositories for the APOLLO Network.}
American Medical Informatics Association Annual Symposium. Washington,
D.C.; 11/6/2017.

Majahalme N, Miller S, Zimolzak A, Vassy J. \textbf{Scaling Down
  Clinical Trial Software from 13,500 to 400 Participants.} American
Medical Informatics Association Annual Symposium. Washington, D.C.;
11/6/2017.

Miller SJ, Hau C, Majahalme N, Zimolzak AJ, MacMullen L, Vassy JL.
\textbf{Potential impact of statin pharmacogenetic testing in an
  integrated healthcare system: The Integrating Pharmacogenetics in
  Clinical Care (I-PICC) Study.} American Society of Human Genetics
Annual Meeting. Orlando; 10/19/2017.

Johnson BR, Fillmore N, Zimolzak A, Ho YL, Elbers D, Katcher B, Gagnon
D, Meng F, Brophy M, Fiore L, Lesse A, Concato J, Gaziano JM, Do N,
Elkin PL, Cho K. \textbf{Developing validated phenotypic cancer
  cohorts for molecular stratification and susceptibility assessment,
  a use case: patients diagnosed with early versus late stage
  non-small cell lung cancer.} American Society of Human Genetics
Annual Meeting. Orlando; 10/19/2017.

Johnson BR, Fillmore NR, Brophy M, Fiore LD, Elkin PL, Katcher B, Ho Y-L,
Zimolzak AJ. \textbf{Phenotyping Non-Small-Cell Lung Cancer at the VA:
  Cohort and Susceptibility Analysis.} VA Research week. Boston;
5/18/2017.

Fillmore NR, Zimolzak AJ, Johnson BR, Brophy M, Munshi N.
\textbf{Ascertaining Cases of Multiple Myeloma Using Multiple VA
  National Structured Data Sources.} VA Research week. Boston;
5/18/2017.

Zimolzak AJ, Kudesia VM. \textbf{Secondary Use of an
  Eighty-Billion-Row Clinical Data Warehouse.} Society for Industrial
and Applied Mathematics 2016 annual meeting. Boston; 7/14/2016.

Raju SP, Ho Y-L, Zimolzak AJ, Katcher B, Cho K, Gagnon DR.
\textbf{Validation of Laboratory Values in a Heterogeneous Healthcare
  System: The US Veterans Affairs Experience.} 31st International
Conference on Pharmacoepidemiology \& Therapeutic Risk Management
(ICPE). Boston; 8/22\ndash{}26/2015.

Leatherman SM, Riley KE, Woods PA, Zimolzak AJ, Majahalme N, Kudesia
V, Ferguson RE, Fiore LD. \textbf{Ascertainment of Clinical Outcomes
  from Electronic Medical Record Data for Point-of-Care Clinical
  Trials.} HSR\&D / QUERI National Conference. Philadelphia;
7/\linebreak[0]8\ndash{}\linebreak[0]10/\linebreak[0]2015.

Zimolzak AJ. \textbf{Early Detection of Statin
  Adherence/Nonadherence.} National Library of Medicine Informatics
Training Conference, Salt Lake City; 6/19/2013.

Zimolzak AJ, Spettell CM, Fernandes J, Fusaro VA, Palmer NP, Saria S,
Jonikas M, Kohane IS, Mandl KD. \textbf{Early Statin Adherence As a
  Predictor of Later Adherence.} American Medical Informatics
Association Summit on Clinical Research Informatics. San Francisco;
3/20/2013.

\section*{Professional Service}

\textbf{Institutional Review Board,} VA Boston Healthcare System.
9/2017\ndash{}11/2018.\\
\textbf{NEJM Group Open Forum.} Moderator for a forum where
authors discuss their NEJM articles. 2/2015\ndash{}3/2015.\\
\textbf{PLoS ONE,} reviewer. 11/2012\ndash{}present.\\
\textbf{Journal of Health Services Research \& Policy,} reviewer.
8/2015\ndash{}present.\\
\textbf{AMIA Annual Symposium,} reviewer. 2015\ndash{}present.\\
\textbf{Physicians and Social Media: Keeping it Professional.}
Planning committee member, panel moderator. 9/26/2013.\\
\textbf{Critical Data.} Planning committee member, and participant.
``Data marathon'' events at MIT. Assisted my team with SQL data pull,
predictive modeling, clinical expertise.
1/3/2014\ndash\linebreak[0]1/7/2014, and 9/5/2014\ndash{}9/7/2014.\\
\textbf{Computing in Cardiology.} Planning committee \& reviewer.
Three day conference for international participants from medicine,
physics, engineering and computer science.
9/8/2014\ndash{}9/10/2014.\\
\textbf{Massachusetts American College of Physicians Annual Scientific
  Meeting.} Planning committee, moderator for \emph{Jeopardy!}-style
quiz tournament, reviewer for abstract/poster competition,
2014\ndash{}2016.

\section*{References}
\begin{tabular}{llll}
Mary Brophy & 150 S.\ Huntington Ave, mail stop 151MAV & Boston MA
02130 & 857-364-4201\\
Nhan Do & 150 S.\ Huntington Ave, mail stop 151MAV & Boston MA
02130 & 617-232-9500\\
Jay Orlander & 1400 VFW Pkwy & West Roxbury MA 02132 & 857-203-5111\\
Anthony Breu & 1400 VFW Pkwy & West Roxbury MA 02132 & 857-203-5111\\
Valmeek Kudesia & 30 Winter St & Boston MA 02108 & 617-726-0600\\
Kenneth Mandl & 300 Longwood Ave & Boston MA 02115 & 617-355-6624\\
David Meenan & 133 Brookline Ave & Boston MA 02215 & 617-421-1194\\
Alexa McCray & 10 Shattuck St & Boston MA 02115 & 617-432-2144\\
Miguel Paniagua & 3750 Market St & Philadelphia PA 19104 & 215-590-9500
\end{tabular}

\end{document}

% LocalWords:  Desloge Shattuck MMSc ent Mandl LM radiographs Meenan Advisor KD
% LocalWords:  handoffs Precepted rotators Paniagua Lumen Schneeweiss Aurel FDT
% LocalWords:  Cami Spettell Fernandes Fusaro Saria Jonikas Kohane Adeimy ra ry
% LocalWords:  Clostridium difficile Clinicopathologic Longwood ly li ing Krak
% LocalWords:  Adherers PLoS advisor HVR hackathon Marantz PMD Celi Surescripts
% LocalWords:  Badawi Feng Ghassemi Ippolito RG Mayaud Naumann Pimentel TJ JMIR
% LocalWords:  Medstro informatician MAVERIC Valmeek Kudesia Fiore BLS Coursera
% LocalWords:  reproducibility MediaWiki Udacity Medtech Libreplanet MAV edX KE
% LocalWords:  Orlander Breu Raju Katcher Cho Gagnon Healthcare ICPE Leatherman
% LocalWords:  Majahalme LD HSR QUERI Grudberg CME clinicopathologic BD myeloma
%%  LocalWords:  Brookline PGY th Informatics SAS Aetna multivariable Nhan WUMS
%%  LocalWords:  hyperlipidemia biostatistics pharmacoepidemiology AJ username
%%  LocalWords:  Alexa nonadherence informatics CPRS VistA Cerner SLU zimolzak
%%  LocalWords:  PowerChart USMLE ACLS Brophy Turek Lavori Pract NEJM NaNoWriMo
%%  LocalWords:  Statins Statin AMIA Danziger Cham Springer electrophoresis De
% LocalWords:  satisfiability antihypertensives CFR llll versa GitHub Peirce JL
%%  LocalWords:  Elkin Phenotyping Munshi DeDomenico Meng Elbers Ajjarapu PODR
% LocalWords:  Selva Vassy Hau MacMullen pharmacogenetic healthcare PICC YL JM
% LocalWords:  Pharmacogenetics Lesse Concato Gaziano phenotypic ICD smartphone
%%  LocalWords:  genomic UMLS Vandana Menon Leavy statin VM FHIR Advani Ayandeh
% LocalWords:  Begley Holcombe IQuESt Translational Hardeep DARPA
%%  LocalWords:  phenotyping analytics pharmacogenetics Contemp Clin
