\documentclass[10pt]{article}
\usepackage{cv} %use xelatex
\title{Curriculum Vitae}
\author{Andrew J.\ Zimolzak}
\begin{document}

\thispagestyle{fancy}

\section*{Personal Information}

\begin{tabular}{l l}
\textbf{Home address:}           & \textbf{Research/clinical address:}\\
1090 Beacon Street, Apartment 2B & 150 S.\ Huntington Ave, 151-MAV\\
Brookline, MA 02446              & Boston, MA 02130\\
517-740-7761 (mobile)            & 857-364-3392\\
andyzimolzak@gmail.com           & Andrew.Zimolzak@va.gov\\
\\
\textbf{Birth date:} November 21, 1980\\
\textbf{Birthplace:} Alma, MI, U.S.A.
\end{tabular}

\section*{Education}

\textbf{Michigan State University.} 8/13/1998\ndash5/3/2002.\\
150 Administration Building; East Lansing, MI 48824-0210.\\
BS in Biochemistry with High Honor. Member of the Honors College.

\textbf{Washington University in St.\ Louis School of Medicine.}
8/19/2002\ndash\linebreak[0]5/6/\linebreak[0]2007.\\
Campus Box 8021, 660 S.\ Euclid; St.\ Louis, MO 63110.\\
MD.

\textbf{University of Missouri-Columbia.} 7/1/2007\ndash6/30/2008.\\
MA419 Medical Sciences Building DC043.00; Columbia, MO 65212.\\
Internal medicine PGY 1.

\textbf{Saint Louis University Hospital.} 7/1/2008\ndash6/30/2010.\\
3635 Vista Avenue, 14th Floor S, Desloge Towers; St.\ Louis MO
63110-0250.\\
Internal medicine PGY 2 and 3.

\textbf{Harvard Medical School.} 7/1/2011\ndash5/30/2013.\\
25 Shattuck Street; Boston, MA 02115.\\
MMSc in Medical Informatics.

\section*{Professional Experience} % reverse chronol.

\textbf{Clinical informatician,} Massachusetts Veterans Epidemiology
Research and Information Center (MAVERIC),
10/20/\linebreak[0]2014\ndash{}present. I am a clinical subject matter
expert who reviews large collections of fragmented data elements to
unify them into a single clinical concept. These elements come from a
data warehouse covering twenty million unique individuals, from 1999
to the present, and aggregated from 130 VA sites. I work with three
main projects: a precision medicine platform and ultimately learning
health system for cancer; a platform for point of care, randomized
comparative effectiveness trials of medications; and the early stages
of a sponsored hepatitis C observational cohort study. For the third
project, I am the principal investigator. I am also developing a
system to ensure that the data cleaning or harmonization efforts of
our research group and other groups are reproducible and reusable
across the whole VA. I work with multiple other VA studies,
translating between clinical investigators and data pull
engineers/analysts or natural language processing experts. Finally, I
work as a teaching attending on the inpatient medicine service and for
a physical diagnosis course. I have supervised house staff and
students from three residency programs, two medical schools, and one
interdisciplinary PhD program. Supervisors (research): Valmeek
Kudesia, Louis Fiore. Supervisors (clinical, educational): Steven
Simon, Jay Orlander, Anthony Breu, Sarah Grudberg.

\textbf{Research collaborator,} Massachusetts Institute of Technology,
Laboratory for Computational Physiology, 12/6/2013\ndash{}\linebreak[0]present.
Lead clinician / subject matter expert on a project to develop a
data-based definition of acute kidney injury, using the clinical
records of 11,000 patients from MIMIC II, a large, open database of
intensive care unit data collected over 8 years. Collaborated on
writing and submission of an NIH R25 grant, pertaining to open
educational resources for biomedical big data, and courses for skills
development in biomedical big data science. Professor: Leo Celi,
Harvard-MIT Division of Health Science and Technology.

\textbf{Research fellow,} Children’s Hospital Informatics Program,
Intelligent Health Lab\-o\-ra\-to\-ry,
\linebreak[0]7/1/\linebreak[0]2011\ndash{}\linebreak[0]1/31/2014.
Analyzed large pharmacy claims databases to develop predictive models
of medication adherence. Used SAS software to operate on the primary
data set from Aetna, Inc., which selected commercial insurance members
with hyperlipidemia and comprised 61 million enrollment records, 200
million medical claims, and 90 million prescription claims. Secondary
data sets included laboratory and health risk assessment data.
Performed modeling using multivariable logistic regression and
multivariable adaptive regression splines. Quantified prescription
refill irregularity using Fourier spectral analysis. Took coursework
including Harvard School of Public Health Program in Clinical
Effectiveness (introductory epidemiology, biostatistics, health
services research, pharmacoepidemiology), biomedical informatics, data
mining \& statistical machine learning, population health informatics,
business \& management, and grant writing. Professor: Kenneth Mandl,
Harvard Medical School. Funded by a National Library of Medicine grant
to program director Alexa McCray, Harvard Medical School.

\textbf{Urgent care physician,} Harvard Vanguard Medical Associates,
10/15/2011\ndash{}present. Diagnose and treat patients with acute
medical problems, including common respiratory and gastrointestinal
infections, sprains \& strains, abscesses, simple lacerations, pelvic
conditions, and initial management of fractures. Supervise and consult
with nurse practitioners and physician assistants as needed. Respond
to “code blue” emergencies in the building. Interpret urgent
radiographs, electrocardiograms, and laboratory tests. Our clinic
capabilities include intravenous fluids and intravenous antibiotics.
Supervisor: David Meenan.

\textbf{External consultant/advisor,} Medication Adherence Request for
Proposals, Aetna, Inc.,
1/\linebreak[0]2013\ndash{}\linebreak[0]7/2013. Reviewed competitive
proposals/bids from 7 vendors to Aetna, Inc., for a contract to
identify modifiable risk factors for medication nonadherence. Provided
expertise on statistical modeling techniques and on published
literature about medication adherence.

\textbf{Fellowship liaison,} Harvard Medical School,
6/2012\ndash5/2013. Coordinated a twice-week\-ly, two-hour informatics
seminar, a for-credit course that 20 students attend. Assisted with
recruiting and scheduling 25 different speakers from Harvard and
externally. Assured group consensus about the direction of this
fellow-run seminar.

\textbf{Early career physician council,} Massachusetts American
College of Physicians,
9/\linebreak[0]2011\ndash{}\linebreak[0]pres\-ent. Organize activities
for early career internists in Massachusetts, as part of an
approximately 15-person council. Sit on planning committee for
continuing medical education events, coordinate social events.

\textbf{Internal medicine chief resident,} Saint Louis University
Hospital, 7/1/2010\ndash\linebreak[0]6/30/\linebreak[0]2011. Managed
and scheduled 130 residents and rotators. Included academic
appointment as instructor in internal medicine and hospital
appointment as physician. Attended on inpatient \& outpatient general
medicine services in university and VA hospitals. Accepted transfers,
supervised and improved patient handoffs. Precepted medical students,
ran approximately 100 resident report conferences, 6 morbidity and
mortality conferences, and two clinicopathologic conferences.
Interviewed residency program applicants and sat on selection
committee. Program director: Miguel Paniagua, Saint Louis University.

\textbf{Research elective,} 2/2007\ndash4/2007. Analyzed associations
in a database of 12,000 outpatients associated with 40,000 diagnoses.
Wrote Perl code to find associations between any given pair of
diseases in a patient, and to output the strongest associations in an
easily visualized format. Professor: Walton Sumner, Washington
University in St.\ Louis.

\textbf{Professorial assistant,} 9/1998\ndash5/2002. Used molecular
biology and cell culture techniques to analyze factors leading to
malignant transformation of human cells. Professor: J.\ Justin
McCormick, Michigan State University.

\textbf{Technology skills.} Experience with SAS and R statistical
programming, Git source code management, and basic SQL all since 2011;
and with the Python programming language since 2013. Experience with
the Perl programming language, Linux/UNIX shell commands and shell
programming, \LaTeX, and GNU Emacs since 2001. Intermediate experience
with audio, video, and photo equipment editing. \emph{Electronic
  medical records:} experience with Epic Hyperspace since 2009,
CPRS/VistA since 2007, and 1 year experience with Cerner PowerChart.

\section*{Credentials and Memberships}

\textbf{American Board of Internal Medicine} certified 8/10/2011,
valid through 12/31/2021.\\
\textbf{Clinical Informatics} board-certified 1/1/2014, valid through
1/31/2024.\\
\textbf{Massachusetts} full medical license 6/2011\ndash{}present.\\
\textbf{Missouri} full medical license 6/2010\ndash{}present.\\
\textbf{American College of Physicians} member 2010\ndash{}present.\\
\textbf{USMLE} passed Step 1 6/2005, Step 2 CS 4/2007, Step 2 CK 4/2007, Step 3
5/2010.\\
\textbf{ACLS} and BLS last certification 11/9/2014.

\section*{Honors and Awards}

\textbf{Finalist,} Surescripts 2014 Adherence Challenge. Member of a
team selected as one of ten finalists. Primary responsibility for
drafting the proposal. 8/15/2014\ndash{}1/8/2015.\\
\textbf{Third prize, patient safety \& quality improvement category,}
8/2010. American College of Physicians, Missouri Chapter Scientific
Meeting, associate poster competition.\\
\textbf{Tower Guard,} 1999\ndash2000. Service-oriented academic honor
society for Michigan State University sophomores.\\
\textbf{Distinguished Freshman Scholarship,} Michigan State
University. Competitive four-year full tuition scholarship awarded to
35 students out of the incoming class of about 10,000.

\section*{Outside Interests}

\textbf{Online courses} in machine learning (Coursera), theoretical
computer science (Udacity), and contract law (edX).\\
\textbf{Literature:} Boston combined residency in pediatrics book
club, 2014. SLU Literature and Medicine Group member 2008, 2010.
National Novel Writing Month winner. Washington University Medical
School Write Club member.\\
\textbf{Class show} technical director, spring 2003, 2004, and 2006.\\
\textbf{Additional interests:} travel (Stockholm, Avignon, Krak\'ow),
black and white photography, social dance, bread baking, Rubik’s Cube,
sailing (Mercury 15$'$, Sonar 23$'$).

\section*{Publications}
Badawi O, Brennan T, Celi LA, Feng M, Ghassemi M, Ippolito A, Johnson
A, Mark RG, Mayaud L, Moody G, Moses C, Naumann T, Pimentel M, Pollard
TJ, Santos M, Stone DJ, Zimolzak A. (2014) \textbf{Making Big Data
  Useful for Health Care: A Summary of the Inaugural MIT Critical Data
  Conference.} JMIR Med Inform 2(2):e22.\\
doi:10.2196/medinform.3447

Celi LA, Zimolzak AJ, Stone DJ. (2014) \textbf{Dynamic Clinical Data
  Mining: Search Engine-Based Decision Support.} JMIR Med Inform
2(1):e13. doi:10.2196/medinform.3110

Zimolzak AJ, Spettell CM, Fernandes J, Fusaro VA, Palmer NP, Saria S,
Kohane IS, Jonikas M, Mandl KD. (2013) \textbf{Early Detection of Poor
  Adherers to Statins: Applying Individualized Surveillance to Pay for
  Performance.} PLoS ONE 8(11):e79611. doi:10.1371/journal.pone.0079611

Zimolzak AJ. \textbf{Medication Adherence: How Should We Measure It,
  and Can We Detect It Early?} Harvard Medical School, Master of
Medical Science thesis, 5/17/2013. Committee members: Kenneth
D.\ Mandl, MD, MPH; Sebastian Schneeweiss, MD, SD; Aurel Cami, PhD.

Zimolzak AJ, Fiore L. \textbf{Precision Medicine: Sounds Good, Right?}
[Internet]. Medtech Boston. 2015.\\
https://medtechboston.medstro.com/precision-medicine-sounds-good-right/

\section*{Presentations} %reverse chron

Raju SP, Ho Y-L, Zimolzak AJ, Katcher B, Cho K, Gagnon DR.
\textbf{Validation of Laboratory Values in a Heterogeneous Healthcare
  System: The US Veterans Affairs Experience.} 31st International
Conference on Pharmacoepidemiology \& Therapeutic Risk Management
(ICPE). Boston; 8/22\ndash{}26/2015.

Leatherman SM, Riley KE, Woods PA, Zimolzak AJ, Majahalme N, Kudesia
V, Ferguson RE, Fiore LD. \textbf{Ascertainment of Clinical Outcomes
  from Electronic Medical Record Data for Point-of-Care Clinical
  Trials.} HSR\&D / QUERI National Conference. Philadelphia;
7/\linebreak[0]8\ndash{}\linebreak[0]10/\linebreak[0]2015.

Zimolzak AJ. \textbf{Early Detection of Statin
  Adherence/Nonadherence.} National Library of Medicine Informatics
Training Conference, Salt Lake City; 6/19/2013.

Zimolzak AJ, Spettell CM, Fernandes J, Fusaro VA, Palmer NP, Saria S,
Jonikas M, Kohane IS, Mandl KD. \textbf{Early Statin Adherence As a
  Predictor of Later Adherence.} American Medical Informatics
Association Summit on Clinical Research Informatics. San Francisco;
3/20/2013.

Adeimy C, Zimolzak AJ, Paniagua MA. \textbf{Improved Supervision of
  Transfers of Care and Education for Night Residents.} Missouri
American College of Physicians meeting, Osage Beach, MO; 8/2010.
American College of Physicians national meeting, San Diego, CA;
4/7/2011.

\section*{Professional Service}

\textbf{NEJM Group Open Forum.} Moderator for a forum where
authors discuss their NEJM articles. 2/2015\ndash{}3/2015.\\
\textbf{Medstro Primary Care Innovation Challenge.} Reviewed proposed
ideas with a panel of judges in an initial round and in a final round
at a live event on 9/16/2014.\\
\textbf{PLoS ONE,} reviewer. 11/2012\ndash{}present.\\
\textbf{Journal of Health Services Research \& Policy,} reviewer.
8/2015\ndash{}present.\\
\textbf{AMIA Annual Symposium,} reviewer. 2015\ndash{}present.\\
\textbf{Physicians and Social Media: Keeping it Professional.} Planning committee member and panel moderator. 9/26/2013.\\
\textbf{Critical Data.} Planning committee member, participant. “Data
marathon” events at MIT. Assisted my team with SQL data pull,
predictive modeling, clinical expertise.
1/3/2014\ndash\linebreak[0]1/7/2014, and 9/5/2014\ndash{}9/7/2014.\\
\textbf{Computing in Cardiology.} Planning committee member, abstract
reviewer, social event planner. Three day conference for international
participants from medicine, physics, engineering and computer science.
9/8/2014\ndash{}9/10/2014.\\
\textbf{Massachusetts American College of Physicians Annual Scientific
  Meeting.} Planning committee member, moderator for
\emph{Jeopardy!}-style quiz tournament for residents, reviewer for resident and student abstract competition, 2014\ndash{}2015.\\
\textbf{Stand Down.} Medical volunteer. Outreach event for homeless
veterans. 8/28/2015.

\section*{References}
Valmeek Kudesia. 150 S.\ Huntington Ave mail stop 151MAV; Boston MA
02130.\\
Louis Fiore. 150 S.\ Huntington Ave mail stop 151MAV; Boston MA
02130.\\
Kenneth Mandl. 300 Longwood Ave; Boston MA 02115. 617-355-6624.\\
David Meenan. 133 Brookline Ave; Boston MA 02215. 617-421-1194.\\
Alexa McCray. 10 Shattuck St; Boston MA 02115. 617-432-2144.\\
Miguel Paniagua. 3750 Market St; Philadelphia PA 19104. 215-590-9500.

\end{document}

% LocalWords:  Desloge Shattuck MMSc ent Mandl LM radiographs Meenan Advisor KD
% LocalWords:  handoffs Precepted rotators Paniagua Lumen Schneeweiss Aurel FDT
% LocalWords:  Cami Spettell Fernandes Fusaro Saria Jonikas Kohane Adeimy ra ry
% LocalWords:  Clostridium difficile Clinicopathologic Longwood ly li ing Krak
% LocalWords:  Adherers PLoS advisor HVR hackathon Marantz PMD Celi Surescripts
% LocalWords:  Badawi Feng Ghassemi Ippolito RG Mayaud Naumann Pimentel TJ JMIR
% LocalWords:  Medstro informatician MAVERIC Valmeek Kudesia Fiore BLS Coursera
% LocalWords:  reproducibility MediaWiki Udacity Medtech Libreplanet MAV edX KE
% LocalWords:  Orlander Breu Raju Katcher Cho Gagnon Healthcare ICPE Leatherman
% LocalWords:  Majahalme LD HSR QUERI Grudberg CME clinicopathologic
