\documentclass[10pt]{article}
\usepackage{cv} %use xelatex
\title{Curriculum Vitae}
\author{Andrew J.\ Zimolzak}
\begin{document}

\thispagestyle{fancy}

\section*{Personal Information}

\begin{tabular}{l l}
\textbf{Home address:}           & \textbf{Research/clinical address:}\\
1090 Beacon Street, Apartment 2B & 150 S.\ Huntington Ave, 151-MAV\\
Brookline, MA 02446              & Boston, MA 02130\\
517-740-7761 (mobile)            & 857-364-3392\\
andyzimolzak@gmail.com           & Andrew.Zimolzak@va.gov\\
\\
\textbf{Birth date:} November 21, 1980\\
\textbf{Birthplace:} Alma, MI, U.S.A.
\end{tabular}

\section*{Education}

\textbf{Michigan State University.} 8/13/1998\ndash5/3/2002.\\
150 Administration Building; East Lansing, MI 48824-0210.\\
BS in Biochemistry with High Honor. Member of the Honors College.

\textbf{Washington University in St.\ Louis School of Medicine.}
8/19/2002\ndash\linebreak[0]5/6/\linebreak[0]2007.\\
Campus Box 8021, 660 S.\ Euclid; St.\ Louis, MO 63110.\\
MD.

\textbf{University of Missouri-Columbia.} 7/1/2007\ndash6/30/2008.\\
MA419 Medical Sciences Building DC043.00; Columbia, MO 65212.\\
Internal medicine PGY 1.

\textbf{Saint Louis University Hospital.} 7/1/2008\ndash6/30/2010.\\
3635 Vista Avenue, 14th Floor S, Desloge Towers; St.\ Louis MO
63110-0250.\\
Internal medicine PGY 2 and 3.

\textbf{Harvard Medical School.} 7/1/2011\ndash5/30/2013.\\
25 Shattuck Street; Boston, MA 02115.\\
MMSc in Medical Informatics.

\section*{Professional Experience} % reverse chronol.

\textbf{Clinical informatician,} Massachusetts Veterans Epidemiology
Research and Information Center (MAVERIC),
10/20/\linebreak[0]2014\ndash{}present. \textbf{Assistant professor,}
Boston University School of Medicine, 8/2016\ndash{}present.

I am a clinical subject matter expert who reviews large collections of
fragmented data elements to unify them into clinically meaningful
concepts, thus making diverse clinical research studies possible.
These efforts are often termed \textbf{secondary use} of electronic
health record data, or \textbf{clinical research informatics.} I often
interpret clinical language and systems for data pull
engineers/analysts, or vice versa (interpreting technical language for
clinical investigators). I access a data warehouse covering twenty
million unique individuals, anbd billions of observations, from 1999
to the present, aggregated from 130 VA sites (but incompletely
standardized/harmonized). My work makes possible projects such as: a
\textbf{genomic precision medicine} platform and ultimately learning
health system for cancer; and a 13,000-subject point of care,
randomized \textbf{comparative effectiveness trial} of two
antihypertensives. I also help map VA data elements to external
repositories' data models so that patients' data may be available to
non-VA researchers with consent, and I am developing a system to
ensure that our data cleaning or harmonization efforts can be
inspected, reproduced, and reused by others.

I am the \textbf{co-program-director} of the Big Data-Scientist
Training Enhancement Program (BD-STEP), for Boston, one of six sites
nationally. I supervise two research fellows who have doctoral
training in nonclinical fields, whose projects will use Million
Veteran Program data to find genomic predictors of multiple myeloma
progression, response or adverse effects of cancer treatment, and
stage of lung cancer at presentation. I am also the research mentor
for an internal medicine intern, who helped me make a highly accurate
natural language processing classifier of serum protein
electrophoresis reports. I have primary responsibility for knowledge
of\mdash{}and trainee compliance with\mdash{}regulations such as 45
CFR 164.514, and the distinction between research and health care
operational use of data. Supervisors (research): Nhan Do, Valmeek
Kudesia, Mary Brophy, Louis Fiore.

Finally, I work as a \textbf{teaching attending} on the inpatient
medicine service and for a physical diagnosis course. I supervise
house staff and students from three residency programs, two medical
schools, and an interdisciplinary PhD program. Supervisors (clinical,
educational): Steven Simon, Jay Orlander, Anthony Breu, Sarah
Grudberg.

\textbf{Research collaborator,} Massachusetts Institute of Technology,
Laboratory for Computational Physiology,
12/6/2013\ndash{}\linebreak[0]present. Lead clinician on a project to
develop a data-based definition of acute kidney injury, using records
of 11,000 patients from MIMIC II, an open collection of intensive care
unit data collected over 8 years. Collaborated on writing and
submission of an NIH R25 grant. Professor: Leo Celi, Harvard-MIT
Division of Health Science and Technology.

\textbf{Research fellow,} Children’s Hospital Informatics Program,
Intelligent Health Lab\-o\-ra\-to\-ry,
\linebreak[0]7/1/\linebreak[0]2011\ndash{}\linebreak[0]1/31/2014.
Analyzed pharmacy claims to develop predictive models of medication
adherence. Used \textbf{SAS software and multivariable logistic
  regression} to analyze 61 million enrollment records, 200 million
medical claims, and 90 million prescription claims. I designed and
wrote all code to perform the analysis, resulting in 2,000 lines of
code in the final project. Other data and methods included laboratory
and health risk assessment data, multivariable adaptive regression
splines, and Fourier spectral analysis. Coursework in
\textbf{epidemiology, biostatistics, and data mining} through Harvard
School of Public Health Program in Clinical Effectiveness, and courses
in machine learning, and grant writing. Professor: Kenneth Mandl,
Harvard Medical School. Program director: Alexa McCray, Harvard
Medical School.

\textbf{Urgent care physician,} Harvard Vanguard Medical Associates,
10/15/2011\ndash{}present. Diagnose and treat patients with acute
medical problems, e.g.\ respiratory or gastrointestinal infections,
dehydration, sprains, abscesses, lacerations, pelvic conditions,
initial management of fractures. Supervise nurse practitioners and
physician assistants. Supervisor: David Meenan.

\textbf{Fellowship liaison,} Harvard Medical School,
6/2012\ndash5/2013. Coordinated a twice-week\-ly, two-hour informatics
seminar, a for-credit course that 20 students attend. Assisted with
recruiting and scheduling 25 different speakers from Harvard and
externally. Assured group consensus about the direction of this
fellow-run seminar.

\textbf{Internal medicine chief resident,} Saint Louis University
Hospital, 7/1/2010\ndash\linebreak[0]6/30/\linebreak[0]2011. Managed
and scheduled 130 residents. Academic appointment as instructor in
internal medicine and hospital appointment as physician. Attended on
inpatient \& outpatient general medicine in university and VA
hospitals. Accepted transfers, supervised and improved patient
handoffs. Precepted medical students, ran approximately 100 resident
report conferences, 6 morbidity and mortality conferences, and two
clinicopathologic conferences. Interviewed residency program
applicants and sat on selection committee. Program director: Miguel
Paniagua, Saint Louis University.

\textbf{Research elective,} 2/2007\ndash4/2007. Analyzed associations
in a database of 12,000 outpatients associated with 40,000 diagnoses.
Wrote \textbf{Perl code} to find associations between any given pair of
diseases in a patient, and to output the strongest associations in an
easily visualized format. Professor: Walton Sumner, Washington
University in St.\ Louis.

\textbf{Professorial assistant,} 9/1998\ndash5/2002. Used molecular
biology and cell culture techniques to analyze factors leading to
malignant transformation of human cells. Professor: J.\ Justin
McCormick, Michigan State University.

\textbf{Technology skills.} Experience with SAS and R statistical
programming, Git source code management, and SQL all since 2011;
and with the Python programming language since 2013. GitHub username:
zimolzak (have written code involving topics such as Twitter, cryptography,
satisfiability solving, celestial navigation, and Amazon Web
Services tasks). Experience with the Perl programming language,
Linux/UNIX shell commands and shell programming, \LaTeX, and GNU Emacs
since 2001. Intermediate experience with audio, video, and photo
equipment and editing. \emph{Electronic medical records:} experience
with Epic Hyperspace since 2009, CPRS/VistA since 2007, and 1 year
experience with Cerner PowerChart.

\section*{Credentials and Memberships}
\textbf{American Board of Internal Medicine} certified 8/10/2011,
valid through 12/31/2021.\\
\textbf{Clinical Informatics} board-certified 1/1/2014, valid through
1/31/2024.\\
\textbf{Massachusetts} full medical license 6/2011\ndash{}present.\\
\textbf{Missouri} full medical license 6/2010\ndash{}present.\\
\textbf{American College of Physicians} member 2010\ndash{}present.\\
\textbf{Early career physician council,} Massachusetts American
College of Physicians, 9/\linebreak[0]2011\ndash{}2015.\\
\textbf{USMLE} passed Step 1 6/2005, Step 2 CS 4/2007, Step 2 CK
4/2007, Step 3 5/2010.\\
\textbf{ACLS} and BLS last certification 11/9/2014.

\section*{Honors and Awards}

\textbf{Finalist,} Surescripts 2014 Adherence Challenge. Member of a
team selected as one of ten finalists. Primary responsibility for
drafting the proposal. 8/15/2014\ndash{}1/8/2015.\\
\textbf{Third prize, patient safety \& quality improvement category,}
8/2010. American College of Physicians, Missouri Chapter Scientific
Meeting, associate poster competition.\\
\textbf{Tower Guard,} 1999\ndash2000. Service-oriented academic honor
society for Michigan State University sophomores.\\
\textbf{Distinguished Freshman Scholarship,} Michigan State
University. Competitive four-year full tuition scholarship awarded to
35 students out of the incoming class of about 10,000.

\section*{Outside Interests}

\textbf{Online courses} in machine learning (Coursera), theoretical
computer science (Udacity), and contract law (edX).\\
\textbf{Literature:} Book clubs with pediatrics residency \& SLU
Medical School. NaNoWriMo winner. WUMS Write Club.\\
\textbf{Class show} technical director, spring 2003, 2004, and 2006.\\
\textbf{Additional interests:} travel (Stockholm, Avignon, Krak\'ow),
black and white photography, Rubik’s Cube, sailing (Mercury 15$'$,
Sonar 23$'$), ukulele.

\section*{Publications}

Danziger J, Zimolzak AJ. \textbf{Residual Confounding Lurking in Big
  Data: A Source of Error.} In: MIT Critical Data, editors. Secondary
Analysis of Electronic Health Records. Cham, Switzerland: Springer;
2016. p.\ 71\ndash{}78.

Fiore L, Ferguson RE, Brophy M, Kudesia V, Shannon C, Zimolzak A,
Pierce-Murray K, Turek S, Lavori P. \textbf{Implementation of a
  Precision Oncology Program as an Exemplar of a Learning Health Care
  System in the VA.} Fed Pract. 2016 February;33(suppl
1):S26\ndash{}S30.

Badawi O, Brennan T, Celi LA, Feng M, Ghassemi M, Ippolito A, Johnson
A, Mark RG, Mayaud L, Moody G, Moses C, Naumann T, Pimentel M, Pollard
TJ, Santos M, Stone DJ, Zimolzak A. (2014) \textbf{Making Big Data
  Useful for Health Care: A Summary of the Inaugural MIT Critical Data
  Conference.} JMIR Med Inform 2(2):e22.\\
doi:10.2196/medinform.3447

Celi LA, Zimolzak AJ, Stone DJ. (2014) \textbf{Dynamic Clinical Data
  Mining: Search Engine-Based Decision Support.} JMIR Med Inform
2(1):e13. doi:10.2196/medinform.3110

Zimolzak AJ, Spettell CM, Fernandes J, Fusaro VA, Palmer NP, Saria S,
Kohane IS, Jonikas M, Mandl KD. (2013) \textbf{Early Detection of Poor
  Adherers to Statins: Applying Individualized Surveillance to Pay for
  Performance.} PLoS ONE 8(11):e79611.
doi:10.1371/journal.pone.0079611

Zimolzak AJ. \textbf{Medication Adherence: How Should We Measure It,
  and Can We Detect It Early?} Harvard Medical School, Master of
Medical Science thesis, 5/17/2013.

\section*{Presentations} %reverse chron

Zimolzak AJ. \textbf{Secondary Use of an Eighty-Billion-Row Clinical
  Data Warehouse.} Society for Industrial and Applied Mathematics 2016
annual meeting. Boston; 7/14/2016.

Raju SP, Ho Y-L, Zimolzak AJ, Katcher B, Cho K, Gagnon DR.
\textbf{Validation of Laboratory Values in a Heterogeneous Healthcare
  System: The US Veterans Affairs Experience.} 31st International
Conference on Pharmacoepidemiology \& Therapeutic Risk Management
(ICPE). Boston; 8/22\ndash{}26/2015.

Leatherman SM, Riley KE, Woods PA, Zimolzak AJ, Majahalme N, Kudesia
V, Ferguson RE, Fiore LD. \textbf{Ascertainment of Clinical Outcomes
  from Electronic Medical Record Data for Point-of-Care Clinical
  Trials.} HSR\&D / QUERI National Conference. Philadelphia;
7/\linebreak[0]8\ndash{}\linebreak[0]10/\linebreak[0]2015.

Zimolzak AJ. \textbf{Early Detection of Statin
  Adherence/Nonadherence.} National Library of Medicine Informatics
Training Conference, Salt Lake City; 6/19/2013.

Zimolzak AJ, Spettell CM, Fernandes J, Fusaro VA, Palmer NP, Saria S,
Jonikas M, Kohane IS, Mandl KD. \textbf{Early Statin Adherence As a
  Predictor of Later Adherence.} American Medical Informatics
Association Summit on Clinical Research Informatics. San Francisco;
3/20/2013.

\section*{Professional Service}

\textbf{NEJM Group Open Forum.} Moderator for a forum where
authors discuss their NEJM articles. 2/2015\ndash{}3/2015.\\
\textbf{PLoS ONE,} reviewer. 11/2012\ndash{}present.\\
\textbf{Journal of Health Services Research \& Policy,} reviewer.
8/2015\ndash{}present.\\
\textbf{AMIA Annual Symposium,} reviewer. 2015\ndash{}present.\\
\textbf{Physicians and Social Media: Keeping it Professional.}
Planning committee member, panel moderator. 9/26/2013.\\
\textbf{Critical Data.} Planning committee member, and participant.
``Data marathon'' events at MIT. Assisted my team with SQL data pull,
predictive modeling, clinical expertise.
1/3/2014\ndash\linebreak[0]1/7/2014, and 9/5/2014\ndash{}9/7/2014.\\
\textbf{Computing in Cardiology.} Planning committee \& reviewer.
Three day conference for international participants from medicine,
physics, engineering and computer science.
9/8/2014\ndash{}9/10/2014.\\
\textbf{Massachusetts American College of Physicians Annual Scientific
  Meeting.} Planning committee, moderator for \emph{Jeopardy!}-style
quiz tournament, reviewer for abstract/poster competition,
2014\ndash{}2016.

\section*{References}
Valmeek Kudesia. 30 Winter St; Boston MA 02130. 617-726-0600.\\
Louis Fiore. 150 S.\ Huntington Ave mail stop 151MAV; Boston MA
02130.\\
Kenneth Mandl. 300 Longwood Ave; Boston MA 02115. 617-355-6624.\\
David Meenan. 133 Brookline Ave; Boston MA 02215. 617-421-1194.\\
Alexa McCray. 10 Shattuck St; Boston MA 02115. 617-432-2144.\\
Miguel Paniagua. 3750 Market St; Philadelphia PA 19104. 215-590-9500.

\end{document}

% LocalWords:  Desloge Shattuck MMSc ent Mandl LM radiographs Meenan Advisor KD
% LocalWords:  handoffs Precepted rotators Paniagua Lumen Schneeweiss Aurel FDT
% LocalWords:  Cami Spettell Fernandes Fusaro Saria Jonikas Kohane Adeimy ra ry
% LocalWords:  Clostridium difficile Clinicopathologic Longwood ly li ing Krak
% LocalWords:  Adherers PLoS advisor HVR hackathon Marantz PMD Celi Surescripts
% LocalWords:  Badawi Feng Ghassemi Ippolito RG Mayaud Naumann Pimentel TJ JMIR
% LocalWords:  Medstro informatician MAVERIC Valmeek Kudesia Fiore BLS Coursera
% LocalWords:  reproducibility MediaWiki Udacity Medtech Libreplanet MAV edX KE
% LocalWords:  Orlander Breu Raju Katcher Cho Gagnon Healthcare ICPE Leatherman
% LocalWords:  Majahalme LD HSR QUERI Grudberg CME clinicopathologic BD myeloma
%%  LocalWords:  Brookline PGY th Informatics SAS Aetna multivariable Nhan WUMS
%%  LocalWords:  hyperlipidemia biostatistics pharmacoepidemiology AJ username
%%  LocalWords:  Alexa nonadherence informatics CPRS VistA Cerner SLU zimolzak
%%  LocalWords:  PowerChart USMLE ACLS Brophy Turek Lavori Pract NEJM NaNoWriMo
%%  LocalWords:  Statins Statin AMIA Danziger Cham Springer electrophoresis
% LocalWords:  satisfiability antihypertensives
