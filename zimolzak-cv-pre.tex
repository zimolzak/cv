\documentclass[10pt]{article}
\usepackage{cv} %use xelatex
\title{Curriculum Vitae}
\author{Andrew J.\ Zimolzak}
\begin{document}

\thispagestyle{fancy}

\section*{Personal Information}

\begin{tabular}{l l}
\textbf{Research address:}  & \textbf{Home address:}\\
x-office    & x-home\\
x-ocity    & x-hcity\\
x-phone    & x-mobile\\
zimolzak@bcm.edu\\
\\
\textbf{Birth date:} x-bday\\
\textbf{Birthplace:} x-bplace
\end{tabular}




\section*{Education}

\textbf{Michigan State University.} 8/13/1998\ndash5/3/2002.\\
150 Administration Building; East Lansing, MI 48824-0210.\\
BS in Biochemistry with High Honor. Member of the Honors College.

\textbf{Washington University in St.\ Louis School of Medicine.}
8/19/2002\ndash\linebreak[0]5/6/\linebreak[0]2007.\\
Campus Box 8021, 660 S.\ Euclid; St.\ Louis, MO 63110.\\
MD.

\textbf{University of Missouri-Columbia.} 7/1/2007\ndash6/30/2008.\\
MA419 Medical Sciences Building DC043.00; Columbia, MO 65212.\\
Internal medicine PGY 1.

\textbf{Saint Louis University Hospital.} 7/1/2008\ndash6/30/2010.\\
3635 Vista Avenue, 14th Floor S, Desloge Towers; St.\ Louis MO
63110-0250.\\
Internal medicine PGY 2 and 3.

\textbf{Harvard Medical School.} 7/1/2011\ndash5/30/2013.\\
25 Shattuck Street; Boston, MA 02115.\\
MMSc in Medical Informatics.




\section*{Professional Experience} % reverse chron

\textbf{Clinical informatician} at the VA Center for Innovations in
Quality, Effectiveness and Safety (IQuESt), and at the Baylor
Institute for Clinical and Translational Research,
11/12/2018\ndash{}present. \textbf{Assistant professor,} Baylor
College of Medicine. I am a clinical subject matter expert who reviews
large collections of fragmented data elements to unify them into
clinically meaningful concepts, thus making diverse clinical research
studies possible.  These efforts are often termed \textbf{secondary
  use} of electronic health record data, or \textbf{clinical research
  informatics.} I interpret clinical language and systems for data
pull engineers/analysts, or vice versa (interpreting technical
language for clinical investigators). I access a data warehouse
covering twenty million unique individuals, and billions of
observations, from 1999 to the present, aggregated from 130 VA sites
(but incompletely standardized/harmonized). Major projects include
disseminating algorithms to detect missed laboratory test follow-ups,
and clinical trials matching. I work as a \textbf{teaching
  attending} on the VA inpatient medicine service, supervising house
staff and students. Supervisors: Hardeep Singh, Laura
Petersen, Chris Amos, Andrew Caruso.

\textbf{Clinical informatician,} Massachusetts Veterans Epidemiology
Research and Information Center (MAVERIC),
10/20/\linebreak[0]2014\ndash{}11/9/2018. \textbf{Assistant
  professor,} Boston University School of Medicine,
8/2016\ndash{}11/2018, and reciprocal appointment as Lecturer, Harvard
Medical School. My work enabled a \textbf{genomic precision medicine}
platform and learning health system for cancer, as well as a
13,000-subject point of care, randomized \textbf{comparative
  effectiveness trial} of two antihypertensives. I developed (2016) a
system to ensure \textbf{reusable phenotyping algorithms} for genomic
association and other studies, and I directly exported and encrypted
data for transfer to the mirror system adopted at Oak Ridge National
Laboratory (2018). I collaborated on writing and submission of a U01
grant. I served as \textbf{co-program-director} of the Big
Data-Scientist Training Enhancement Program (BD-STEP) site in Boston,
and served as principal investigator on a study of genomic predictors
of stage of lung cancer at presentation. I mentored an internal
medicine resident, who helped me make a highly accurate
\textbf{natural language processing classifier} of serum protein
electrophoresis reports. Inpatient medicine \textbf{teaching
  attending} at the VA hospital. Supervisors (research): Nhan Do,
Valmeek Kudesia, Mary Brophy, Louis Fiore. Clinical: Steven Simon, Jay
Orlander, Anthony Breu.

\textbf{Research collaborator,} Massachusetts Institute of Technology,
Laboratory for Computational Physiology,
12/6/2013\ndash{}\linebreak[0]9/2016. Lead clinician on a project to
develop a data-based definition of acute kidney injury, using records
of 11,000 patients from MIMIC II, an open collection of intensive care
unit data collected over 8 years. Collaborated on writing and
submission of an NIH R25 grant. Professor: Leo Celi, Harvard-MIT
Division of Health Science and Technology.

\textbf{Research fellow,} Children's Hospital Informatics Program,
Intelligent Health Lab\-o\-ra\-to\-ry,
\linebreak[0]7/1/\linebreak[0]2011\ndash{}\linebreak[0]1/31/2014.
Analyzed pharmacy claims to develop predictive models of medication
adherence. Used \textbf{SAS software and multivariable logistic
  regression} to analyze 61 million enrollment records, 200 million
medical claims, and 90 million prescription claims. I designed and
wrote all code to perform the analysis, resulting in 2,000 lines of
code in the final project. Other data and methods included laboratory
and health risk assessment data, multivariable adaptive regression
splines, and Fourier spectral analysis. Coursework in
\textbf{epidemiology, biostatistics, data mining, machine learning,
  and grant writing,} partially through Harvard School of Public
Health Program in Clinical Effectiveness. Professor: Kenneth Mandl,
Harvard Medical School. Program director: Alexa McCray, Harvard
Medical School.

\textbf{Urgent care physician,} Harvard Vanguard Medical Associates,
10/15/2011\ndash{}6/30/2018. Diagnose and treat patients with acute
medical problems, e.g.\ respiratory or gastrointestinal infections,
dehydration, sprains, abscesses, lacerations, pelvic conditions,
initial management of fractures. Supervise nurse practitioners and
physician assistants. Supervisor: David Meenan.

\textbf{Internal medicine chief resident,} Saint Louis University
Hospital, 7/1/2010\ndash\linebreak[0]6/30/\linebreak[0]2011. Managed
and scheduled 130 residents. Academic appointment as instructor in
internal medicine and hospital appointment as physician. Attended on
inpatient \& outpatient general medicine in university and VA
hospitals. Accepted transfers, supervised and improved patient
handoffs. Precepted medical students, ran approximately 100 resident
report conferences, 6 morbidity and mortality conferences, and two
clinicopathologic conferences. Interviewed residency program
applicants and sat on selection committee. Program director: Miguel
Paniagua, Saint Louis University.

\textbf{Research elective,} 2/2007\ndash4/2007. Analyzed associations
in a database of 12,000 outpatients associated with 40,000 diagnoses.
Wrote Perl code to find associations between any given pair of
diseases in a patient, and to output the strongest associations in an
easily visualized format. Professor: Walton Sumner, Washington
University in St.\ Louis.

\textbf{Professorial assistant,} 9/1998\ndash5/2002. Used molecular
biology and cell culture techniques to analyze factors leading to
malignant transformation of human cells. Professor: J.\ Justin
McCormick, Michigan State University.

\textbf{Technology skills.} Experience with \textbf{SAS} and
\textbf{R} statistical programming, \textbf{Git} source code
management, and \textbf{SQL} all since 2011; and with the
\textbf{Python} programming language since 2013. GitHub username:
zimolzak (have written code involving topics such as a Twitter text
generator, cryptography, satisfiability solving, celestial navigation,
and Amazon Web Services tasks). Experience with the Perl programming
language, Linux/UNIX, \LaTeX, and GNU Emacs since 2001. Intermediate
experience with audio, video, and photo equipment and editing.
\emph{Electronic medical records:} experience with Epic Hyperspace
since 2009, CPRS/VistA since 2007, and 1 year experience with Cerner
PowerChart.

\textbf{Additional courses.} \emph{Machine Learning}
(Coursera/Stanford, Andrew Ng, Octave language, 2011). \emph{Intro to
Theoretical Computer Science} (Udacity, Sebastian Wernicke, Python
language, 2014). \emph{Tackling the Challenges of Big Data} (edX/MIT,
Daniela Rus, Sam Madden, Michael Stonebraker,
2/3/2015\ndash{}3/17/2015). \emph{Data Science Boot Camp} (Rice
University, in-person, Chris Jermaine, Devika Subramanian: numpy,
Jupyter, scikit-learn, 2019, 5 days). \emph{Machine Learning with
Python: From Linear Models to Deep Learning} (edX/MIT 6.86x, Regina
Barzilay, Tommi Jaakkola: \textbf{numpy and pytorch} for RNNs, CNNs,
etc., 2020, 11 weeks. Took 9600 words of notes with extensive
mathematical notation and wrote around 2000 lines of code. In-class
exercises average 95\%, projects average 100\%.)




\section*{Grants}

Co-Investigator: \textbf{An Electronic Trigger Tool to Detect Missed
  Opportunities for Barrett’s Esophagus Screening.} Baylor College of
Medicine, Department of Medicine Physician-Scientist/Investigator
Faculty Development Award. 07/01/2022\ndash{}06/30/2024. Amount:
\$207,906.

Co-Investigator: \textbf{Houston Patient Safety Center of Inquiry:
  Diagnosis Improvement Safety Center (DISCovery).} VHA National
Center for Patient Safety. FY 2022\ndash{}FY 2024 (3 years). Amount:
\$1,185,000.

% fixme - awaiting funding decisions. uncomment if funded.

%% Co-Investigator: \textbf{Diagnostic Safety Center for Advancing
%%   E-triggers and Rapid Feedback Implementation (DISCOVERI).} AHRQ R18.
%% 04/01/2023\ndash{}03/31/2027. Amount: \$3,999,998. Drafted a large
%% portion of the proposal and managed multiple steps in integrating
%% sections, completing sections, and finalizing with administrators.

%% Co-Investigator: \textbf{Measuring Missed Opportunities in the
%%   Diagnosis of Gastrointestinal Cancers.} Gordon and Betty Moore Foundation.
%% INSERT DATES. GBMF NNNN. Amount: INSERT for 18 months.

%% Advisor: \textbf{Identifying current clinical care processes to
%% reduce delays in colorectal cancer diagnoses.} DATES. FOUNDATION.
%% AMOUNT.

Co-PI: \textbf{Safer Dx e-measures to reduce preventable delays in
  cancer diagnosis.} Gordon and Betty Moore Foundation.
11/2019\ndash{}5/2021. GBMF 8838. Amount: \$520,162 for 18 months.
Drafted the majority of the proposal.

Co-Investigator: \textbf{Application of a Machine Learning to Enhance
  e-Triggers to Detect and Learn from Diagnostic Safety Events.} AHRQ.
09/30/19\ndash{}09/29/2022. R01 HS27363-01. Amount: \$496,121.

Co-Investigator: \textbf{The Safer Dx Learning Lab: A Demonstration
  Project for Improving Diagnostic Safety.} Gordon and Betty Moore
Foundation. 6/30/2017\ndash{}5/31/2020. GBMF 5498. Amount: \$3,525,397
for 36 months.




\section*{Mentoring relationships}
% Reverse chronological. "Name (formality). Date. Description."

\textbf{Disha Kumar} (informal). 7/2022\ndash{}present. Chief resident
in quality and safety at VA hospital. Mentored in data retrieval and
phenotyping strategies to identify inpatients with advanced cirrhosis,
and missed follow-up of colorectal cancer screening.

\textbf{Max Yu} (informal). 12/2021\ndash{}present. Computer science
undergraduate coming on board to assist in analysis. Mentored in
clinical topics, VA data environment, and regulatory compliance.

\textbf{Theresa Nguyen} (formal mentoring). 9/2021\ndash{}present.
Fellow awarded Department of Medicine grant that includes me on
mentorship team. Mentoring in VA data availability and predictive
model application.

\textbf{Paarth Kapadia} (informal). 1/2020\ndash{}present. MD student
at Baylor College of Medicine doing year-long research experience.
Mentored in manuscript preparation, data analysis, regulatory
compliance.

\textbf{Nathanael Fillmore} (formal mentoring). 2016\ndash{}2017.
Nonclinical PhD accepted to a VA advanced fellowship. As co-program
director for BD-STEP, I was responsible for training all fellows
regarding clinical topics, the data environment, and regulatory
compliance. Hired full-time by VA in 2017, informal mentoring
thereafter. Dr.\ Fillmore was later promoted to director of machine
learning at the Boston VA clinical trials center.

\textbf{Brett Johnson} (formal mentoring). 2016\ndash{}2017. BD-STEP
fellow. Responsibilities: see \emph{Fillmore} above. Hired full-time
by VA in 2017, informal mentoring thereafter.

\textbf{Justine Ryu} (formal \& informal). 7/2016 research elective,
through 2020 publication. Internal medicine resident at Boston
University. Mentored in exposure to machine learning techniques,
journal submission, and revision.

\textbf{Sahar Alkhairy} (informal). 2014\ndash{}2020. Undergraduate
and graduate student in computer science and molecular biology at MIT.
Mentored in research record keeping, clinical topics, journal
submission, and revision.




\section*{Credentials and Memberships}
\textbf{American Board of Internal Medicine} certified 8/10/2011,
valid through 12/31/2022 (MOC deadline extended).\\
\textbf{Clinical Informatics} board-certified 1/1/2014, valid through
1/31/2024.\\
\textbf{Texas} full medical license 8/2018\ndash{}present. No.\ R8850,
exp 8/31/2024.\\
\textbf{Massachusetts} full medical license 6/2011\ndash{}present.
No.\ 249050 (inactive), exp 11/21/2023.\\
\textbf{Missouri} full medical license 6/2010\ndash{}present.
No.\ 2010020878, exp 1/31/2023\\
\textbf{American College of Physicians} member 2010\ndash{}present.\\
\textbf{American Medical Informatics Association} member
2011\ndash{}present.\\
\textbf{Early career physician council,} Massachusetts American
College of Physicians, 9/\linebreak[0]2011\ndash{}2015.\\
\textbf{USMLE} passed Step 1 6/2005, Step 2 CS 4/2007, Step 2 CK
4/2007, Step 3 5/2010.\\
\textbf{ACLS} and BLS certified.




\section*{Honors and Awards}

\textbf{Best of the best oral abstracts,} Society to Improve Diagnosis
in Medicine, 10/2022. Title: ``Implementation, validation, and
mortality association of 2 cancer diagnosis digital quality
measures.''\\
\textbf{Most innovative use of data,} Baylor College of Medicine
Datathon, April 2022. Project entitled ``The association of biased
language in medical text with diagnostic error.''\\
\textbf{Excellence in Collaboration,} Baylor College of Medicine
Datathon, October 2020. Project entitled ``Understanding predictors of
severe COVID-19 outcomes.'' One of the few projects to achieve
end-to-end success in identifying data elements, data extraction,
analysis, and results. All analyses performed by me in essentially 24
hours.\\
\textbf{Reviewers' Choice Abstract,} American Society of Human
Genetics Annual Meeting, 2017. Abstract entitled ``Developing
validated phenotypic cancer cohorts\ldots{}.''\\
\textbf{Finalist,} Surescripts 2014 Adherence Challenge. Member of a
team selected as one of ten finalists. Primary responsibility for
drafting the proposal. 8/15/2014\ndash{}1/8/2015.\\
\textbf{Third prize, patient safety \& quality improvement category,}
8/2010. American College of Physicians, Missouri Chapter Scientific
Meeting, associate poster competition.\\
\textbf{Tower Guard,} 1999\ndash2000. Service-oriented academic honor
society for Michigan State University sophomores.\\
\textbf{Distinguished Freshman Scholarship,} Michigan State
University. Competitive four-year full tuition scholarship awarded to
35 students out of the incoming class of about 10,000.




\section*{Publications}

% Format:
% Authors. (year) Title. Journal vol(iss):pages.

Zimolzak AJ, \emph{et al.} \textbf{Lessons Learned from an
  Enterprise-Wide Clinical Datathon.} \emph{Under revision, August
2022.}

Meyer AND, Singh H, Zimolzak AJ, Wei L, Choi DT, Marinez A, Murphy DR.
\textbf{Evaluation for cancer during the COVID-19 pandemic: an
  observational study using national veterans affairs electronic
  health record data.} American Journal of Preventive Medicine.
In-press journal pre-proof August 8, 2022. 10.1016/j.amepre.2022.07.004

Zimolzak AJ, Singh H, Murphy DR, \emph{et al.} (2022)
\textbf{Translating electronic health record-based patient safety
  algorithms from research to clinical practice at multiple sites.}
BMJ Health Care Inform 29(1):e100565.

Alkhairy S, Celi LA, Feng M, Zimolzak AJ. (2021) \textbf{Acute Kidney
  Injury Detection Using Refined and Phys\-i\-o\-log\-i\-cal-Fea\-ture
  Augmented Urine Output.} Scientific Reports 11(1):19561.

Zimolzak AJ, Shahid U, Giardina T, Memon S, Mushtaq U, Zubkoff L,
Murphy DR, Bradford A, Singh H. (2022) \textbf{Why Test Results Are
  Still Getting ``Lost'' to Follow-up: A Qualitative Study of
  Implementation Gaps.} Journal of General Internal Medicine
37(1):137\ndash{}144. Published online 2021-04-27.

Vassy JL, Gaziano JM, Green RC, Ferguson RE, Advani S, Miller SJ, Chun
S, Hage AK, Seo SJ, Majahalme N, MacMullen L, Zimolzak AJ, Brunette
CA. (2020) \textbf{Effect of Pharmacogenetic Testing for Statin
  Myopathy Risk vs Usual Care on Blood Cholesterol: A Randomized
  Clinical Trial.} JAMA Network Open 3(12):e2027092.

Ryu JH, Zimolzak AJ. (2020) \textbf{Natural Language Processing of
  Serum Protein Electrophoresis Reports in the Veterans Affairs Health
  Care System.} JCO Clinical Cancer Informatics 4:749\ndash{}756.

Brunette CA, Miller SJ, Majahalme N, Hau C, MacMullen L, Advani S,
Ludin SA, Zimolzak AJ, Vassy JL. (2020) \textbf{Pragmatic Trials in
  Genomic Medicine: The Integrating Pharmacogenetics in Clinical Care
  (I-PICC) Study.} Clinical and Translational Science
13(2):381\ndash{}390. Epub 2019-12-18.

Fillmore N, Do N, Brophy M, Zimolzak A. (2019) \textbf{Interactive
  Machine Learning for Laboratory Data Integration.} Stud Health
Technol Inform 264:133\ndash{}137.

Fillmore NR, Yellapragada SV, Ifeorah C, Mehta A, Cirstea D, White PS,
Rivero G, Zimolzak A, Pyarajan S, Do N, Brophy M, Munshi NC. (2019)
\textbf{With equal access, African Americans have superior survival
  compared to Caucasians with Multiple Myeloma: a VA study.} Blood
133(24):2615\ndash{}2618.

Vassy JL, Brunette CA, Majahalme N, Advani S, MacMullen L, Hau C,
Zimolzak AJ, Miller SJ. (2018) \textbf{The Integrating
  Pharmacogenetics in Clinical Care (I-PICC) Study: Protocol for a
  point-of-care randomized controlled trial of statin pharmacogenetics
  in primary care.} Contemp Clin Trials 75:40\ndash{}50.

Danziger J, Zimolzak AJ. \textbf{Residual Confounding Lurking in Big
  Data: A Source of Error.} In: MIT Critical Data, editors. Secondary
Analysis of Electronic Health Records. Cham, Switzerland: Springer;
2016. pp.\ 71\ndash{}78.

Fiore L, Ferguson RE, Brophy M, Kudesia V, Shannon C, Zimolzak A,
Pierce-Murray K, Turek S, Lavori P. (2016) \textbf{Implementation of a
  Precision Oncology Program as an Exemplar of a Learning Health Care
  System in the VA.} Fed Pract 33(suppl 1):S26\ndash{}S30.

Badawi O, Brennan T, Celi LA, Feng M, Ghassemi M, Ippolito A, Johnson
A, Mark RG, Mayaud L, Moody G, Moses C, Naumann T, Pimentel M, Pollard
TJ, Santos M, Stone DJ, Zimolzak A. (2014) \textbf{Making Big Data
  Useful for Health Care: A Summary of the Inaugural MIT Critical Data
  Conference.} JMIR Med Inform 2(2):e22.

Celi LA, Zimolzak AJ, Stone DJ. (2014) \textbf{Dynamic Clinical Data
  Mining: Search Engine-Based Decision Support.} JMIR Med Inform
2(1):e13.

Zimolzak AJ, Spettell CM, Fernandes J, Fusaro VA, Palmer NP, Saria S,
Kohane IS, Jonikas M, Mandl KD. (2013) \textbf{Early Detection of Poor
  Adherers to Statins: Applying Individualized Surveillance to Pay for
  Performance.} PLoS ONE 8(11):e79611.

Zimolzak AJ. \textbf{Medication Adherence: How Should We Measure It,
  and Can We Detect It Early?} Harvard Medical School, Master of
Medical Science thesis, 5/17/2013.




\section*{Presentations} %reverse chron

Zimolzak AJ, Kapadia P, Murphy DR, Upadhyay D, Mushtaq U, Mir U,
Offner A, Korukonda S, Murugaesh Rekha R, Abel G, Lyratzopoulos G,
Mounce L, Singh H. \textbf{Development/Implementation of Cancer
  Diagnosis Digital Quality Measures.} American Medical Informatics
Association Annual Symposium, selected for oral presentation.
Washington DC, 11/2022.

Peng FB, Kumar D, Vaclavik L, Horstman MJ, Zimolzak AJ, Jackson LK,
Braun UK, Hernaez R, Flores AG. \textbf{Quality improvement in
  palliative hepatology: increasing curative and palliative care for
  veterans with decompensated cirrhosis.} The Liver Meeting (AASLD,
poster). Washington DC, 11/2022.

Zimolzak AJ, Kapadia P, Murphy DR, Upadhyay D, Mushtaq U, Mir U,
Offner A, Korukonda S, Murugaesh Rekha R, Abel G, Lyratzopoulos G,
Mounce L, Singh H. \textbf{Implementation, validation, and mortality
  association of 2 cancer diagnosis digital quality measures.} Society
to Improve Diagnosis in Medicine conference (Best of the best oral
presentation). Minneapolis, 10/2022.

Zimolzak AJ, Choi D, Dawson D, Fletcher T, Scott T, Giardina T.
\textbf{The association of race and ethnicity with negative
  descriptors in clinical texts.} Society to Improve Diagnosis in
Medicine conference (oral presentation). Minneapolis, 10/2022.

% above here: SIDM, SIDM, AMIA, Liver Meeting

Zimolzak AJ. \textbf{Concept embeddings for stroke diagnostic error.}
Gulf Coast Consortia Artificial Intelligence in Healthcare Symposium.
5/18/2022.

% begin amia cic

Zimolzak AJ, Singh H, Murphy DR, Wei L, Memon SA, Upadhyay D,
Korukonda S, Zubkoff L, Sittig DF. \textbf{Translating electronic
  health record-based patient safety research algorithms to multiple
  clinical sites.} American Medical Informatics Association Clinical
Informatics Conference, selected for oral presentation. Houston,
5/25/2022.

Murphy DR, Zimolzak AJ, Wei L, Jolly P, Offner A, Sittig DF, Singh H.
\textbf{Developing Digital Quality Measures to Assess Potential Missed
  Opportunities in Cancer Diagnosis.} American Medical Informatics
Association Clinical Informatics Conference. Houston, 5/25/2022.

Kapadia P, Zimolzak AJ, Murphy DR, Upadhyay D, Mushtaq U, Mir U,
Korukonda S, Murugaesh Rekha R, Wei L, Lyratzopoulos G, Abel G, Offner
A, Singh H. \textbf{A Digital Quality Measure of Emergency Cancer
  Diagnosis Using EHR Data in Two Large Health Systems.} American
Medical Informatics Association Clinical Informatics Conference,
selected for oral presentation. Houston, 5/25/2022.

% end amia cic

Zimolzak AJ. \textbf{Translating Electronic Health Record-Based
  Patient Safety Algorithms from Research to Clinical Practice.}
Invited seminar talk, Baylor College of Medicine, Computational and
Integrative Biomedical Research Center. Quantitative \& Computational
Biosciences graduate seminar. 2/23/2022.

% begin datathon (amia, apha)

Zimolzak AJ, Davila JA, Punugoti V, Sippel KH, Balasubramanyam A,
Klotman P, Petersen LA, Rochat RH, Liao G, Laubscher RR, Leiber L,
Amos CI. \textbf{Lessons Learned from an Enterprise-Wide Clinical
  Datathon.} American Medical Informatics Association Annual
Symposium, selected for oral presentation. San Diego, 11/2/2021.

Zimolzak AJ, Davila JA, Amos CI. \textbf{Lessons learned from an
  enterprise-wide clinical datathon.} APHA 2021 Annual Meeting,
selected for oral presentation. Denver, 10/25/2021.

% end datathon
% begin sidm 2021

Zimolzak AJ, Murphy DR, Upadhyay D, Wei L, Mushtaq U, Jolly P,
Korukonda S, Lyratzopoulos G, Abel G, Offner A, Singh H.
\textbf{Development of an electronic quality measure of late stage
  cancer diagnosis.} Society to Improve Diagnosis in Medicine
conference (poster). 10/25/2021.

Meyer AND, Singh H, Zimolzak AJ, Wei L, Marinez A, Murphy DR.
\textbf{Association Between the COVID-19 Pandemic and Follow-up of
  Cancer-Related Abnormal Test Results.} Society to Improve Diagnosis
in Medicine conference. 10/2021.

Kapadia P, Zimolzak AJ, Murphy DR, Upadhyay D, Mushtaq U, Jolly P,
Korukonda S, Wei L, Lyratzopoulos G, Abel G, Offner A, Singh H.
\textbf{Development of an electronic diagnostic quality measure based
  on emergency cancer presentations in the United States.} Society to
Improve Diagnosis in Medicine conference, selected for oral
presentation. 10/2021.

Zimolzak AJ. \textbf{Data processes and machine learning for health
  research, in the US Department of Veterans Affairs.} Invited seminar
talk, University of Houston Department of Industrial Engineering.
4/16/2021.

% begin sidm 2020

Kapadia P, Zimolzak A, Murphy D, Lyratzopoulos G, Abel G, Upadhyay D,
Scott T, Singh H. \textbf{Development of eMeasures to Study Missed and
  Delayed Diagnosis of Lung and Colorectal Cancer.} Society to Improve
Diagnosis in Medicine conference. 10/19/2020.

Vaghani V, Murphy D, Memon S, Zimolzak A, Subramanian D, Upadhyay D,
Singh H. \textbf{A Portfolio of e-Triggers to Identify Diagnostic
  Errors in Emergency Departments: A Prioritization Exercise.} Society
to Improve Diagnosis in Medicine conference. 10/19/2020.

Kapadia P, Vaghani V, Zimolzak A, Singh H, Subramanian D.
\textbf{Development of a Machine Learning Enhanced Trigger to Detect
  Diagnostic Error.} Society to Improve Diagnosis in Medicine
conference. 10/20/2020.

Kapadia P, Murphy D, Zimolzak A, Lyratzopoulos G, Abel G, Scott T,
Singh H. \textbf{Measuring the Proportion of Cancer Diagnosed as an
  Emergency in the VA Health System.} Society to Improve Diagnosis in
Medicine conference. 10/20/2020.

% end sidm 2020

Vaghani V, Wei L, Mushtaq U, Zimolzak A, Sittig D, Singh H.
\textbf{Performance of an e-Trigger to Detect Missed Stroke Diagnosis
  in Patients with Headache or Dizziness Symptoms in Emergency
  Department.} American Medical Informatics
Association Annual Symposium. 11/16/2020.

% end amia 2020
% begin sidm 2019

Zimolzak A, Shahid U, Memon S, Giardina T, Mushtaq U, Murphy D,
Zubkoff L, Singh H. \textbf{Top Contributors to Missed Test Results in
  the VA Health Care System.} Diagnostic Error in Medicine 12th Annual
International Conference. Washington, D.C.; 11/11/2019.

Vaghani V, Wei L, Mushtaq U, Zimolzak A, Sittig D, Singh H.
\textbf{Performance of an e-Trigger to Detect Missed Stroke Diagnosis
  In Patients with Headache or Dizziness Symptoms in Emergency.}
Diagnostic Error in Medicine 12th Annual International Conference.
Washington, D.C.; 11/11/2019.

% end sidm 2019
% begin older boston presentations

Brunette CA, Advani S, Hage A, Seo S-J, Miller SJ, Majahalme N,
Zimolzak AJ, Vassy JL. \textbf{Impact of pharmacogenetic testing on
  statin outcomes: Primary results from the Integrating
  Pharmacogenetics in Clinical Care (I-PICC) Study randomized trial.}
American Society of Human Genetics Annual Meeting. Houston; October
19, 2019.

Do NV, Ramos JC, Fillmore NR, Grossman RL, Fitzsimons M, Elbers DC,
Meng F, Zimolzak AJ, Johnson BR, Ajjarapu S, DeDomenico CL,
Pierce-Murray KE, Hall RB, Do AF, Gaynor K, Elkin PL, Brophy MT.
\textbf{Machine learning methods to predict lung cancer survival using
  the Veterans Affairs Research Precision Oncology Data Commons.}
Medinfo (World Congress of Medical Informatics). Lyon, France;
8/26/2019.

Zimolzak A. \textbf{Early and late stage lung cancer: correlating
  clinical data and genomics in the Million Veteran Program.} American
Thoracic Society International Conference. Dallas; 5/20/2019.

Do N, Bono J, Fillmore N, Zimolzak A, Johnson B, Meng F, Elbers D,
Hall R, Ajjarapu S, Brophy M, Elkin P. \textbf{Development of an AI
  empowered Electronic Molecular Tumor Board Application Connected
  Utilizing the SMART on FHIR Framework.} American Medical Informatics
Association Annual Symposium. San Francisco; 11/5/2018.

Advani S, Brunette CA, Miller SJ, Majahalme N, MacMullen L, Hau C,
Zimolzak AJ, Vassy JL. \textbf{The Integrating Pharmacogenetics in
  Clinical Care (I-PICC) study: Baseline characteristics of
  participants in a point-of-care randomized trial.} American Society
of Human Genetics Annual Meeting. San Diego; 10/17/2018.

Johnson BR, Elbers DC, Pierce E, Meng F, Fillmore N, Ayandeh S, Chen
D, Selva L, Begley NB, Zimolzak AJ, Brophy MT, Do NV. \textbf{The
  Veterans Health Administration's Research Precision Oncology
  Project: Integrating Real-World Data into a Learning Health System.}
American Society of Human Genetics Annual Meeting. San Diego;
10/18/2018.

Brunette CA, Miller SJ, Majahalme N, Hau C, MacMullen L, Advani S,
Zimolzak AJ, Vassy JL. \textbf{Pragmatism in Pharmacogenetics Trials:
  A PRECIS-2 perspective on the Integrating Pharmacogenetics in
  Clinical Care (I-PICC) Study.} American Society of Human Genetics
Annual Meeting. San Diego; 10/19/2018.

Do N, Pierce-Murray K, Peirce E, DeDomenico C, Meng F, Elbers D,
Katcher B, Hall R, Zimolzak A, Ajjarapu S, Shannon C, Turek S, Johnson
B, Fillmore N, Brophy M, Fiore L. \textbf{The Development of the
  Research Precision Oncology Program Data Repository (PODR) in the
  Veterans Affairs Healthcare System.} American Medical Informatics
Association Annual Symposium. Washington, D.C.; 11/6/2017.

Ajjarapu S, Meng F, Elbers D, Do N, Hall R, Pierce-Murray K, Selva L,
Katcher B, Johnson B, Zimolzak A, DeDomenico C, Brophy M, Fiore L.
\textbf{Releasing De-identified Clinical, Imaging, and Genomic Data
  from the VA to External Repositories for the APOLLO Network.}
American Medical Informatics Association Annual Symposium. Washington,
D.C.; 11/6/2017.

Majahalme N, Miller S, Zimolzak A, Vassy J. \textbf{Scaling Down
  Clinical Trial Software from 13,500 to 400 Participants.} American
Medical Informatics Association Annual Symposium. Washington, D.C.;
11/6/2017.

Miller SJ, Hau C, Majahalme N, Zimolzak AJ, MacMullen L, Vassy JL.
\textbf{Potential impact of statin pharmacogenetic testing in an
  integrated healthcare system: The Integrating Pharmacogenetics in
  Clinical Care (I-PICC) Study.} American Society of Human Genetics
Annual Meeting. Orlando; 10/19/2017.

Johnson BR, Fillmore N, Zimolzak A, Ho YL, Elbers D, Katcher B, Gagnon
D, Meng F, Brophy M, Fiore L, Lesse A, Concato J, Gaziano JM, Do N,
Elkin PL, Cho K. \textbf{Developing validated phenotypic cancer
  cohorts for molecular stratification and susceptibility assessment,
  a use case: patients diagnosed with early versus late stage
  non-small cell lung cancer.} American Society of Human Genetics
Annual Meeting. Orlando; 10/19/2017.

Johnson BR, Fillmore NR, Brophy M, Fiore LD, Elkin PL, Katcher B, Ho
Y-L, Zimolzak AJ. \textbf{Phenotyping Non-Small-Cell Lung Cancer at
  the VA: Cohort and Susceptibility Analysis.} VA Research week.
Boston; 5/18/2017.

Fillmore NR, Zimolzak AJ, Johnson BR, Brophy M, Munshi N.
\textbf{Ascertaining Cases of Multiple Myeloma Using Multiple VA
  National Structured Data Sources.} VA Research week. Boston;
5/18/2017.

Zimolzak AJ, Kudesia VM. \textbf{Secondary Use of an
  Eighty-Billion-Row Clinical Data Warehouse.} Society for Industrial
and Applied Mathematics 2016 annual meeting. Boston; 7/14/2016.

Raju SP, Ho Y-L, Zimolzak AJ, Katcher B, Cho K, Gagnon DR.
\textbf{Validation of Laboratory Values in a Heterogeneous Healthcare
  System: The US Veterans Affairs Experience.} 31st International
Conference on Pharmacoepidemiology \& Therapeutic Risk Management
(ICPE). Boston; 8/22\ndash{}26/2015.

Leatherman SM, Riley KE, Woods PA, Zimolzak AJ, Majahalme N, Kudesia
V, Ferguson RE, Fiore LD. \textbf{Ascertainment of Clinical Outcomes
  from Electronic Medical Record Data for Point-of-Care Clinical
  Trials.} HSR\&D / QUERI National Conference. Philadelphia;
7/\linebreak[0]8\ndash{}\linebreak[0]10/\linebreak[0]2015.

Zimolzak AJ. \textbf{Early Detection of Statin
  Adherence/Nonadherence.} National Library of Medicine Informatics
Training Conference, Salt Lake City; 6/19/2013.

Zimolzak AJ, Spettell CM, Fernandes J, Fusaro VA, Palmer NP, Saria S,
Jonikas M, Kohane IS, Mandl KD. \textbf{Early Statin Adherence As a
  Predictor of Later Adherence.} American Medical Informatics
Association Summit on Clinical Research Informatics. San Francisco;
3/20/2013.




\section*{Professional Service}

\textbf{Electronic laboratory notebook working group,} Baylor College
of Medicine. 10/2021\ndash{}7/2022.\\
\textbf{Research Data Security Committee,} IQuESt.
2/2022\ndash{}present.\\
\textbf{House Staff Research Symposium,} reviewer. BCM department of
medicine. 4/2022.\\
\textbf{Lung Precision Oncology Project,} grant review section. VA
Central Office  (CSR\&D). 9/2020.\\
\textbf{Methods Analytics and Data Education Initiative,} IQuESt.
1/2020\ndash{}present.\\
\textbf{Patient, Physician, and Society} preceptor, Baylor College of
Medicine.\\
\textbf{Institutional Review Board,} member, VA Boston Healthcare System.
9/2017\ndash{}11/2018.\\
\textbf{NEJM Group Open Forum.} Moderator for a forum where
authors discuss their NEJM articles. 2/2015\ndash{}3/2015.\\
\textbf{F1000Research,} reviewer. 3/2021\ndash{}present. See also
https://f1000research.com/articles/9-1186\\
\textbf{JCO Clinical Cancer Informatics,} reviewer. 2/2020\ndash{}present.\\
\textbf{Digital Health,} reviewer. 12/2019\ndash{}present.\\
\textbf{JMIR Research Protocols,} reviewer. 1/2019\ndash{}present.\\
\textbf{PLoS ONE,} reviewer. 11/2012\ndash{}present.\\
\textbf{Journal of Health Services Research \& Policy,} reviewer.
8/2015\ndash{}present.\\
\textbf{AMIA Annual Symposium,} reviewer. 2015\ndash{}present.\\
\textbf{Physicians and Social Media: Keeping it Professional.}
Planning committee member, panel moderator. 9/26/2013.\\
\textbf{Critical Data.} Planning committee member, and participant.
``Data marathon'' events at MIT. Assisted my team with SQL data pull,
predictive modeling, clinical expertise.
1/3/2014\ndash\linebreak[0]1/7/2014, and 9/5/2014\ndash{}9/7/2014.\\
\textbf{Computing in Cardiology.} Planning committee \& reviewer.
Three day conference for international participants from medicine,
physics, engineering and computer science.
9/8/2014\ndash{}9/10/2014.\\
\textbf{Massachusetts American College of Physicians Annual Scientific
  Meeting.} Planning committee, moderator for \emph{Jeopardy!}-style
quiz tournament, reviewer for abstract/poster competition,
2014\ndash{}2016.




\section*{References}
\begin{tabular}{llll}

Hardeep Singh & 2002 Holcombe Blvd              & Houston TX 77030 & x-singh\\
Chris Amos    & One Baylor Plaza                & Houston TX 77030 & x-amos\\
Andrew Caruso & 2002 Holcombe Blvd Mailcode 111 & Houston TX 77030 & x-caruso\\

Mary Brophy & 150 S.\ Huntington, mail stop 151MAV & Boston MA 02130 & x-brophy\\
Nhan Do     & 150 S.\ Huntington, mail stop 151MAV & Boston MA 02130 & x-do\\

Jay Orlander    & 1400 VFW Pkw      & West Roxbury MA 02132 & x-wrox\\
Anthony Breu    & 1400 VFW Pkw      & West Roxbury MA 02132 & x-wrox\\
Valmeek Kudesia & 30 Winter St      & Boston MA 02108       & x-kudesia\\
Kenneth Mandl   & 300 Longwood Ave  & Boston MA 02115       & x-mandl\\
David Meenan    & 133 Brookline Ave & Boston MA 02215       & x-meenan\\
Alexa McCray    & 10 Shattuck St    & Boston MA 02115       & x-mccray\\
Miguel Paniagua & 3750 Market St    & Philadelphia PA 19104 & x-paniagua
\end{tabular}

\end{document}

% LocalWords:  ocity hcity bday bplace PGY Desloge Shattuck MAVERIC
% LocalWords:  BD Nhan Valmeek Kudesia Brophy Fiore Orlander Breu ra
% LocalWords:  Celi ry SAS Mandl Alexa Meenan handoffs Precepted edX
% LocalWords:  clinicopathologic Paniagua zimolzak PowerChart Udacity
% LocalWords:  Coursera Daniela Rus Stonebraker Devika Subramanian Yu
% LocalWords:  Barzilay Tommi Jaakkola pytorch RNNs CNNs DISCovery AJ
% LocalWords:  GBMF Kapadia Nathanael Ryu Alkhairy MOC phenotypic DT
% LocalWords:  Surescripts Choi Marinez Fea ture Shahid Giardina JL
% LocalWords:  Mushtaq Vassy Gaziano JM Advani Chun Hage Seo Statin
% LocalWords:  Majahalme MacMullen Pharmacogenetic Myopathy JH JCO SV
% LocalWords:  Hau Ludin Pharmacogenetics Epub Technol Yellapragada
% LocalWords:  Ifeorah Mehta Cirstea Rivero Pyarajan Munshi Contemp
% LocalWords:  Clin Danziger Cham Springer Turek Lavori Pract Badawi
% LocalWords:  Ghassemi Ippolito RG Mayaud Naumann Pimentel TJ JMIR
% LocalWords:  Spettell Fernandes Fusaro Saria Kohane Jonikas KD PLoS
% LocalWords:  Adherers Statins Upadhyay Offner Korukonda Murugaesh
% LocalWords:  Rekha Lyratzopoulos Mounce Peng FB Kumar Vaclavik MJ
% LocalWords:  Horstman LK Hernaez AASLD DF Biosciences JA Punugoti
% LocalWords:  Sippel KH Balasubramanyam Klotman Rochat Liao Leiber
% LocalWords:  Laubscher APHA eMeasures Vaghani JC Grossman RL Elbers
% LocalWords:  Fitzsimons Meng Ajjarapu DeDomenico KE Gaynor Elkin De
% LocalWords:  Medinfo Ayandeh Selva Begley Peirce Katcher PODR YL LD
% LocalWords:  Gagnon Lesse Concato Cho VM Raju Pharmacoepidemiology
% LocalWords:  ICPE Leatherman Nonadherence llll singh amos Mailcode
% LocalWords:  caruso MAV brophy Pkw wrox kudesia Longwood mandl
% LocalWords:  meenan mccray paniagua
